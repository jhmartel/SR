\documentclass[12pt]{amsart}
\usepackage{mathrsfs}
\usepackage{amsmath}
\usepackage{amssymb}
\usepackage{amsfonts}
\usepackage{amsopn}
\usepackage{amsthm}
\usepackage{latexsym}
\usepackage[all]{xy}
\usepackage{enumerate}
\usepackage{geometry}
%\usepackage{biblatex}
%\usepackage{hyperref}
%\usepackage[autostyle]{csquotes}
\usepackage{fancyhdr}
\usepackage{graphicx}
\usepackage{wrapfig}
\usepackage{float}

\usepackage[
    backend=biber,
style=alphabetic,
sorting=nyt,
   % style=authoryear-icomp,
    %sortlocale=de_DE,
    %natbib=true,
    %author=true,
%style=verbose,
%journal=true,
%url=true, 
%    doi=false,
%    eprint=true
]{biblatex}
\addbibresource{biblio.bib}

\usepackage[]{hyperref}
\hypersetup{
    colorlinks=true,
}


\newtheorem*{thm}{Theorem}
\newtheorem*{half}{Halfspace Condition}
%\newtheorem{lem}[thm]{Lemma}
%\newtheorem{prop}[thm]{Proposition}
\newtheorem*{prob}{Problem}
%\newtheorem{cor}[thm]{Corollary}
%\newtheorem{question}[thm]{Question}
\newtheorem*{flashbang}{FlashBang Principle}
\newtheorem*{Lefschetz}{Lefschetz theorem}
\newtheorem*{hyp}{Hypothesis}
\theoremstyle{definition}
%\newtheorem{dfn}[thm]{Definition}
%\newtheorem{exx}[thm]{Example}
\theoremstyle{remark}
%\newtheorem{rem}[thm]{Remark}
\newcommand{\fh}{\mathfrak{h}}
\newcommand{\bn}{\mathbf{n}}
\newcommand{\bC}{\mathbb{C}}
\newcommand{\bG}{\mathbb{G}}
\newcommand{\bR}{\mathbb{R}}
\newcommand{\fB}{\mathfrak{B}}
\newcommand{\bZ}{\mathbb{Z}}
\newcommand{\bQ}{\mathbb{Q}}
\newcommand{\bH}{\mathbb{H}}
\newcommand{\bq}{\bar{Q}[t]}
\newcommand{\bb}{\bullet}
\newcommand{\del}{\partial}
\newcommand{\sB}{\mathscr{B}}
\newcommand{\sE}{\mathscr{E}}
\newcommand{\mR}{\bR^\times_{>0}}
\newcommand{\conv}{\overbar{conv}}
\newcommand{\sub}{\del^c \psi^c(x')}
\newcommand{\subb}{\del^c \psi^c(x'')}
\newcommand{\hh}{\hookleftarrow}
\newcommand{\bD}{\mathbb{D}}
\newcommand{\Gm}{\mathbb{G}_m}
\newcommand{\uF}{\underline{F}}
\newcommand{\sC}{\mathscr{C}}
\newcommand{\bX}{\overline{X}^{BS/\bQ}}
\newcommand{\sT}{\mathscr{T}}
\newcommand{\sW}{\mathscr{W}}
\newcommand{\sZ}{\mathscr{Z}}

\begin{document}

%\title{On Einstein's Alleged Proof of Invariance of Spherical Light Waves in Special Relativity}
\title{Lorentz Non Invariance of Spherical Lightwaves in Special Relativity}


\author{J. H. Martel}
\date{\today}
\email{jhmartel@protonmail.com}
\maketitle

\begin{abstract}
The subject of this note is Einstein's alleged proof of the Lorentz invariance of spherical light waves in his special relativity theory. Our purpose is to plainly demonstrate significant errors in the arguments of the above alleged proof. Briefly, the error is made clear by observing that the spatial quantity called ``radius" is not a Lorentz invariant variable. Equivalently, there is no proper Lorentz invariant class of ``radial" solutions to the homogeneous wave equation. Consequently the hypothesis that wave fronts generated by light pulses are spherical in every inertial frame is untenable, and even disproven. This is controversial subject already developed by other critical authors, e.g. \cite{bryant}, \cite{crothers}. The present article arises from the author's own study of the controversy, and his attempt to identify the errors in plain mathematical terms. The consequences of this error on modern physics is not here discussed.

\end{abstract}

\tableofcontents

\section{Admitted Incompatibilities and Attempted Resolutions}

In Einstein's presentation of the basic principles of Special Relativity \cite{einstein2019relativity}, \cite{einstein1905electrodynamics}, there appears an alleged proof of the Lorentz invariance of spherical lightwaves. The purpose of the proof is to reconcile the fundamental assumptions of special relativity, namely:
\begin{itemize}

\item[(A1)] that the laws of physics are the same in all nonaccelerated reference frames, i.e. if $K'$ is a coordinate system moving uniformly (and devoid of rotation) with respect to a coordinate system $K$, then natural phenomena run their course with respect to $K'$ according to exactly the same laws as with respect to $K$. 

\item[(A2)] that light in vacuum propagates along straight lines with constant velocity $c\approx 300,000$ kilometres per second.

\end{itemize}

In Einstein's own words \cite[Ch.7, 11]{einstein2019relativity}:

\begin{quote} 
``\emph{There is hardly a simpler law in physics than that according to which light is propagated in empty space. Every child at school knows, or believes he knows, that this propagation takes place in straight lines with velocity $c=300, 000 km/sec$.\ldots Who would imagine that this simple law has plunged the conscientiously thoughtful physicist into the greatest intellectual difficulties?"} 
\end{quote}

Let the reader observe the reference to ``law" in the above quote. Indeed Einstein presents (A2) as the \emph{law} of propagation of light. Accordingly if (A2) is to be consistent with (A1), then (A2) must necessarily hold true in \emph{every} inertial reference frame. However the conjunction of (A1) and (A2), abbreviated (A12), appears contradictory to the laws of classical mechanics, Galilean transformations, Fizeau's law of addition of velocities, etc.. However Einstein claims that this is only an \emph{apparent} incompatibility. The incompatibility is reconciled, according to Einstein's proposal, by postulating Lorentz-Fitzgerald's length contractions and time dilations, c.f. \cite[Ch.XIV]{michelson}. What needs be demonstrated is that the formulae of the Lorentz transformations preserves the form of the law (A2) in every inertial frame. This requires the Lorentz invariance of luminal spherical waves, as we now discuss.

%Recently the author has learned of the work of Tombe [ref]. 

%In section [ref] we respond to some possible objections, and submit that these two errors invalidate Einstein's argument. 

%The Michelson-Morley experiment demonstrated a null effect: there was no time-variation in the interferometer, no detectable variation with respect to the aether.


%and reconciled by postulating Lorentz contractions and time dilations according to the gamma factor $\gamma=\sqrt{1+v^2/c^2}$.

%Einstein's error in the first steps of his theory has been elaborated by other authors, notably \cite{bryant}, \cite{crothers}. The present note arises from the author's own study of the controversy, and his attempt to identify the incompatibility in plain mathematical terms. In section [ref] we respond to some possible objections, and submit that these two errors invalidate Einstein's argument. 

\section{Lorentz Invariant and Non Invariant Tensors}\label{li}

The Lorentz transformations attempt to account for the observed null effect of Michelson-Morley's famous experiment. The transformations are supposed to relate the space and time coordinates $(x,y,z,t)$ and $(\xi, \eta, \zeta, \tau)$ of two inertial observers $K$ and $K'$, respectively. Formally one assumes Minkowski's line element $h:=dx^2+dy^2+dz^2-c^2dt^2$ is a scalar invariant for all inertial observers, and therefore invariant with respect to Lorentz transformations. Here $c$ is the constant luminal velocity posited by (A2) in vacuum. We warn the reader against casually setting $c=1$ and treating $t$ as a space variable immediately comparable to $x,y,z$. The constant $c$, whether its numerical value is $1$ or not (and with respect to which units?) is necessary to transition from time units to space units. If light satifies (A2), then in a reference frame $K$, light is something $\gamma(t)=(x(t),y(t),z(t))$ that travels through space with time, and whose velocity, if it could be materially measured, would satisfy $$(dx/dt)^2+(dy/dt)^2+(dz/dt)^2=c^2.$$ And in this sense it is argued that light trajectories are constrained to the null cone $N=\{h=0\}$ of Minkowski's metric $h$. Obviously $N$ is Lorentz invariant and satisfies the equation $x^2+y^2+z^2=c^2 t^2$. 

If we fix a reference frame $K$, then the set of Lorentz transformations becomes the Lie group $G:=O(h)$ of isometries of the Minkowski form $h=x^2+y^2+z^2-c^2t^2$. Invariance says $\xi^2+\eta^2+\zeta^2-c^2 \tau^2$ is numerically equal to $x^2+y^2+z^2-c^2 t^2$ for every Lorentz transform $\lambda$ satisfying $(\xi, \eta, \zeta, \tau)=\lambda \cdot (x,y,z,t)$.  It can be shown that Minkowski's form is the \emph{only} Lorentz invariant quadratic form on $\bR^4$ modulo rescaling, c.f. \cite{elton2010indefinite}, \cite{arminjon2018lorentz}.

%Apples and oranges. Cannot drop the constant c and treat x,t as comparable quantities.


%The inertial observers $K, K', \ldots$ of $\bR^3$ are those reference frames, where any pair $K, K'$ of inertial observers are assumed to have a constant relative velocity. This defines the set of inertial frames in bijection with the group of affine motions $\text{Aff}(\bR^3)$ of $\bR^3$. (Formally, the inertial frames are an example of a principal $\text{Aff}(\bR^3)$-space). [Ref] 





Now we turn to our critical analysis. We claim the positive gap in Einstein's attempted proof has a twofold source. 

\textbf{Firstly, an error arises when quadratic expressions like \begin{equation}\label{a}\xi^2+\eta^2+\zeta^2=c^2 \tau^2
\end{equation} are \emph{misidentified} with ``the equation of a sphere".} Strictly speaking, \eqref{a} is a three-dimensional cone in the four variables $\xi, \eta, \zeta, \tau$. Of course the cone contains many spherical two-dimensional subsets, but a further equation is required. For instance the standard round sphere $S$ centred at the origin simultaneously satisfies \eqref{a} and additionally the equation $$\frac{1}{2}d(\xi^2+\eta^2+\zeta^2)=\xi d\xi+\eta d\eta +\zeta d\zeta=0.$$ In short, the round sphere requires that \emph{two} quadratic forms $\xi^2+\eta^2+\zeta^2$ and $c^2\tau^2$ be \emph{simultaneously constant}. 

This leads us to Einstein's second error, which is the failure to observe that \textbf{the Lorentz invariance of the quadratic form $h=x^2+y^2+z^2-c^2t^2$ in no way implies the Lorentz invariance of the forms $h_1=x^2+y^2+z^2$ and $h_2=c^2 t^2.$ } Indeed the quadratic forms $h_1. h_2$ are degenerate, with nontrivial radicals satisfying $$rad(h_1)=\{x=y=z=0\}$$ and $$rad(h_2)=\{t=0\}.$$ The radicals are linear subspaces of $\bR^4$. But if we assume $h_1$ and $h_2$ are Lorentz invariant, then $rad(h_1)$ and $rad(h_2)$ are also invariant. Except the Lorentz group is well known to act irreducibly in its standard representation on $\bR^4$, leading to contradiction.

%The argument is general and applies to split orthogonal groups of arbitrary rank, e.g. $G=SO(n,1)$,$n=1,2,3,\ldots$. %In the most basic case of $SO(1,1)$, the key observation is that a quadratic expression like $x^2=r^2$ in the $xr$-variables does not represent a ``sphere"; the expression represents a sphere of radius $r$ and only if the additional equation $\del x /\del r =0$ holds for every $x$. But the expression $\del x / \del r$ is not a tensor, i.e. satisfies no invariant 

%\section{}



%Now returning to (A12), we look with Einstein to the wave fronts generated by light pulses. 







%We include the following result, essentially due to J.H. Elton. For notation, let $\sC$ be the vector space of possibly degenerate quadratic forms $q$ on $\bR^4$, and let $h\in \sC$ be Minkowski's form. 

%\begin{thm}
%If $q\in \sC$ is a quadratic form on $\bR^4$ such that the restriction $q|_N$ is $G$-invariant, then $q$ is proportional to $h$. 
%\end{thm} 
%\begin{proof}
%The $G$-action on $N$ is transitive on nonzero vectors. Therefore if $q|_N$ is $G$-invariant, then $q|_N$ is constant. By continuity it follows that $q|_N$ is identically zero since $q(0)=0$. So the zero locus of $q$ contains the zero locus of $h$, i.e. \begin{equation}\label{c} N\subset \{q=0\}.
%\end{equation}

%Now we use a theorem of J. H. Elton to conclude $q$, $h$ are proportional, c.f. \cite{elton2010indefinite}. We outline his argument. Let $q\otimes_\bR \bC$ and $h\otimes_\bR \bC$ be the complexifications of the real quadratic forms $q,h$. Thus $q\otimes_\bR \bC: \bR^4 \otimes_\bR \bC \to \bC$ is a complex-valued quadratic form. Elton's proof establishes the following inclusion \begin{equation}\label{b}
%\{h\otimes \bC=0\} \subset \{q\otimes \bC=0\}.
%\end{equation}
%According to the tensor construction we have $h\otimes \bC (x+iy)=h(x)-h(y) + 2i h(x,y)$. Therefore $h\otimes \bC(x+iy)=0$ if and only if $h(x)=h(y)$ and $h(x,y)=0$, where $h(\cdot, \cdot)$ is the bilinear form canonically defined by $h$. Elton's proof reduces to establishing the implication: if \eqref{c} is satisfied, then $h(x)=h(y)$ and $h(x,y)=0$ implies $q(x)=q(y)$ and $q(x,y)=0$ for all $x,y\in \bR^4$. That $q$ vanishes on the null cone $N$ implies $q$  is indefinite if it is not identically zero. Once the inclusion \eqref{b} is established, Hilbert's Nullstellensatz \cite{eisenbud2013commutative} implies $q\otimes \bC=\lambda \cdot h\otimes \bC$ for some $\lambda \in \bC$. But then obviously $\lambda\in \bR$ and the theorem follows.
%We have three cases supposing $h\otimes \bC(x+iy)=0$.
%Case 1: if $h(x)=0$, then $h(y)=0$. But then $h(x+y)=h(x)+h(y)+2h(x,y)=0$ according to the assumptions, and it follows that $q\otimes \bC(x+iy)=0$. 
%Case 2: if $h(x)>0$, then rescaling we can choose $h(x)=1$. Since $h$ is indefinite, there exists $z$ such that $h(z)<0$
%Therefore $h\otimes \bC (x+iy)=0$ if and only if $h(x)=h(y)$ and $h(x,y)=0$. But $h(x-y)=h(x)+h(y)-2h(x,y) 
%\end{proof}

 %This means there exists only one canonical quadratic form with respect to Lorentz transformations.
%That is, the Minkowski form is the unique Lorentz invariant quadratic form (modulo scalars) on $\bR^4$ which vanishes on the null cone. Another argument is possible \cite{arminjon2018lorentz} which establishes nonexistence of any other Lorentz invariant $(0,2)$-tensors (the proof is long algebraic computation). Naturally the question arises whether there exists any other invariant $(p,q)$-tensors which are independant of Minkowski's form $h$.  

Now we return to the subject at hand, namely Einstein's alleged proof that (A12) are compatible with respect to Lorentz transformations. The argument is general, and we need only consider the two-dimensional case in the variables $(x,t)$ and $(\xi, \tau)$. Here we find $h=x^2-c^2 t^2$ is invariant with respect to the group $G=SO(1,1)_0$ generated by $a_\theta:=\begin{pmatrix} \cosh \theta & \sinh \theta \\
\sinh \theta & \cosh \theta
\end{pmatrix}$, where $\theta\in \bR$. The unit ``sphere" includes two vectors 
$\begin{pmatrix} 1 \\ 1\end{pmatrix}$ and $\begin{pmatrix} -1 \\ 1\end{pmatrix}$, and which are mapped by $a_\theta$ to 

$$\begin{pmatrix} \xi \\ \tau \end{pmatrix}=\begin{pmatrix} \cosh \theta+\sinh \theta \\ \cosh \theta+\sinh \theta \end{pmatrix} \text{~~and~~} \begin{pmatrix} -\cosh \theta+\sinh \theta \\ \cosh \theta-\sinh \theta \end{pmatrix}.$$

%$\langle 1, 1 \rangle,~~ \langle -1,1\rangle,$ and which are mapped by $a_\theta \in SO(1,1) \simeq \bR^\times_{>0}$ to $$\langle \xi,\tau\rangle=\langle \cosh \theta+\sinh \theta, \cosh \theta+\sinh \theta \rangle,~~ \langle -\cosh \theta+\sinh \theta, \cosh \theta-\sinh \theta \rangle.$$ 

But evidently $\xi^2 \neq x^2=1$ and $\tau^2 \neq t^2=1$ when $\theta\neq 0$. Thus the quadratic forms $h_1=x^2$ and $h_2=c^2t^2$ are not invariant. Likewise we find the image of the unit sphere $x^2=1$ does not correspond to a sphere in $\xi \tau$ -coordinates. These trivial computations have the nontrivial effect of falsifying the alleged Lorentz invariance of spherical lightwaves.


%In other words the equation $x^2 = c^2 t^2$ is not the equation of sphere when $x,t$ are variable, and likewise $\xi^2=c^2 \tau^2$ is not the equation of sphere when $\xi,\tau$ are both variable and nonconstant. 
%Thus we find a gap in Einstein's purported proof. %The curious reader may consult Einstein's own words, e.g. \cite[Ch.11, pp.39]{einstein2019relativity} and verify that Einstein does not treat the general case but restricts himself to a velocity parallel to $x$-axis.  

%Moreover the algebraic expression \begin{equation}\label{a}\xi^2+\eta^2+\zeta^2=c^2 \tau^2\end{equation} is presumed to represent the ``equation of a sphere". However, the algebraic expression \eqref{a} of four independant variables $\xi, \eta, \zeta,\tau$ does \emph{not} represent a geometric sphere unless, e.g., $$\del \tau / \del \xi=0, ~ \del \tau / \del \eta=0, \del \tau / \del \zeta=0$$ for every $\xi, \eta, \zeta, \tau$ satisfying \eqref{a}. As the above two-dimensional case illustrates, this is not the case.

%Thus we find a positive gap in Einstein's argument, c.f. \cite{bryant}, \cite{crothers}. 
\section{Radius is Not a Lorentz Invariant Variable}\label{wes}
The present section provides another view in terms of the homogeneous wave equation.
 
\begin{equation} \label{hw}
\phi_{xx}+\phi_{yy}+\phi_{zz}-\frac{1}{c^2}\phi_{tt}=0.
\end{equation} 
If $\lambda$ is a Lorentz transformation with $(\xi, \eta, \zeta, \tau)=\lambda \cdot (x,y,z,t)$, then $\phi':=\lambda^*\phi=\phi\circ\lambda^{-1}$ is again a solution of the homogeneous wave equation $$\phi_{\xi \xi}+\phi_{\eta \eta}+\phi_{\zeta \zeta}-\frac{1}{c^2}\phi_{\tau \tau}=0.$$ This is verified by elementary computation, substituting the formulae for the Lorentz transform. \textbf{However the set of \emph{radial solutions} of \eqref{hw} is \emph{not} Lorentz invariant, for indeed there is no Lorentz invariant definition of ``radius".} 

Recall in group theoretic terms, a solution $\phi=\phi(x,t)$ of \eqref{hw} is said to be \emph{radial by an observer $K$} iff $\phi(Ax,t)=\phi(x,t)$ for every rigid motion $A\in SO(3)$ in the space variables $x,y,z$ of $K$. Implicitly this requires a Lie group representation $\rho$ of $SO(3)$ into the Lorentz group $G\simeq O(3,1)$. But this representation of the maximal compact subgroup is noncanonical and cannot be invariantly chosen. Different inertial observers $K, K'$ generally choose different orthogonal symmetry groups, for instance as defined by their own ``physical sum of squares" formulae, applied to physical squares $\xi^2$, $\eta^2$, $\zeta^2$ in their local variables $\xi, \eta, \zeta, \tau$. Of course the Minkowski element \eqref{a} is invariant and canonical, but any decomposition into ``spatial" and ``time" requires arbitrary choices by the observer, and again is not Lorentz invariant. For example, while the open set of timelike vectors $\{h(v)<0\}$ is invariantly defined, there is no Lorentz invariant choice of timelike vector. Likewise among the spacelike set $\{h(v)>0\}$ there is no Lorentz invariant choice of orthogonal three-dimensional frame. Thus there is no Lorentz invariant ``radius". Furthermore, for the motion of light pulses according to (A12) the Minkowski element vanishes identically, and the only canonical tensor element becomes the constant zero element. After a Lorentz change of variables , we find a new solution $\phi'$ as above, but this solution need not be radial in the inertial frame $K'$ . Indeed the rigid $K$-space motion $A$ does not often preserve the space coordinates $\xi, \eta, \zeta$ of $K'$. That is, $A$-motions nontrivially depend on the $K'$-time variable $\tau$. This again reflects the nonexistence of a Lorentz invariant radius.

%Change of variables formula would require the $G$-conjugate $\rho^g$, $g\in G$, 







\section{Some Objections and Responses}
Given the controversial nature of this article, we here respond to some popular objections. 

First, one might object that our argument reduces to the observation that spheres in the frame $K$ are transformed to ellipsoids in $K'$, as well-known \cite[\S 4]{einstein1905electrodynamics}. But we remind the reader that the Lorentz contraction is assumed to affect \emph{material} objects, even independantly of the nature of the material, and such that material spheres in $K$ are seen as material ellipsoids in $K'$, where again the eccentricity of the $K'$-ellipsoid is nontrivial and independant of the material nature of the sphere. But we respond that light spheres are \emph{not} material, and not themselves subject to Lorentz contraction if (A2) holds. Or at least not without further evidence and hypotheses. %Evenmore (A2) clearly requires light waves be spherical in every reference frame if they are spherical in one reference frame.

Secondly, critics may object that (A12) only requires the consistent \emph{measurement} of $c$ in arbitrary reference frames $K, K'$. This would replace the formal ``law" (A2) with a more pragmatic rule of thumb for measurements. This article welcomes such an approach, and which quickly leads us to an important experimental difficulty at the core of special relativity. For we remind the reader that space and time measurements are always dependant on material objects, and often non local, i.e. the source and receivers are possibly separated by large distances. The impossibiliy of synchronizing non local clocks leads to the apparent impossibility of measuring the ``one-way" velocity of light. It even strikes the author that the incompatibility of (A12) is not merely apparent, but \emph{essential} and evidence that (A2) is not a proper natural law. That all measurements of $c$ only succeed in measuring the ``two-way" or ``round-trip" velocities of light where source and receiver coincide, is discussed in \cite{zhang1997special}, \cite{israel}. See also \cite{vid} for entertaining introduction. Thus it appears that (A2) has never been and cannot be subject to measurement. Moreover in studying the two-way velocity of light, one must postulate that the velocity $c$ is constant throughout its journey, as Einstein himself supposed, \cite[Ch.8]{einstein2019relativity}. This is however only an expedient assumption and unverifiable and unfalsfiable by physical experiment.

Thirdly, the interesting textbook \cite[pp.8-10, 21-22]{rindler} attempts 
\begin{quote}
\emph{``in spite of its historical and heuristic importance, $\ldots$ to de-emphasize the logical role of the law of light propagation [(A2)] as a pillar of special relativity."}
\end{quote}
Rindler adds further that
\begin{quote}
\emph{``a second axiom is needed \emph{only} to determine the value of a constant $c$ of the dimensions of a velocity that occurs naturally in the theory. But this could come from any number of branches of physics -- we need only think of the energy formula $E=mc^2$, or de Broglie's velocity relation $u v =c^2$." }
\end{quote}


The above objection is very interesting, and to which we have a simple response, namely that the above quoted formulas are \emph{equivalent} to (A2), and not independant in any logical or physical sense. It should be observed that the constant $c$ was first formulated and estimated by W. Weber circa 1846, and before even J.C. Maxwell published his treatise. It appears that Kirchoff also independantly conceived of the constant $c$ while deriving the telegraphy equations. Here $c$ arises as the velocity along which an electric signal is propagated through a wire of arbitrarily small resistance. Yet the definitions of $c$ are not equivalent -- for Einstein defines $c$ as the velocity in vacuum, and Weber-Kirchoff define $c$ as velocity of propagation in a material wire! See \cite{assis1999meaning} and \cite{awk} for further references and discussion. Even the incredible Weber-Kohlrausch formula expressing $c$ as ratio of electric and magnetic dielectric constants presumes a material medium, i.e. the ratio is undefined in vacuum. While an independant law involving $c$ (velocity in vacuum) could potentially serve as a logical substitute for (A2), this remains strictly hypothetical since no such formula appears to exist -- thus far all formulas involving $c$ are based essentially on some form of (A2). And so the logical pillar remains unmoved. 

A fourth objection might criticize our argument for not properly accounting for the so-called ``wave-particle duality" of light. We respond that our article treats both cases (corpuscular and undulatory), showing that (A12) is false in both cases. In section \ref{wes} we observe that ``radial solutions" of the homogeneous wave equation is not a Lorentz invariant set. That is, there exists no solutions $\phi$ which are radial in every inertial frame. The photon theory is treated in section \ref{li}, see the second paragraph for example. The Minkowski spacetime formalism is basically a form of photon hypothesis. And Einstein's phrasing of (A2) suggests an implicit photon model  of light, especially the assumption that light ``in vacuum travels along straight lines". 

The incompatibility of (A12) with \emph{both} the wave and particle model has been highlighted by A.K.T. Assis \cite[\S 7.2.4, pp.133]{assis1999relational}:  
\begin{quote} \emph{"we can only conclude that for Einstein the velocity of light is constant not only whatever the state of motion of the emitting body [source], but also whatever the state of motion of the receiving body (detector) and of the observer."} 
\end{quote}
For waves in physical medium, the velocity of emission is independant of the velocity of the source. However for both particles and waves, it is also known that the velocity of the wave is dependant on the velocity of the receiver. According to (A2), light then exhibits properties quite unlike both waves and particles. In this sense (A2) contradicts the supposed wave-particle duality. See [Ibid] for further references. 

%A popular fifth objection is that special relativity has apparently been experimentally verified, ad nauseum, a typical example being the clocks on GPS satellites. However the synchronization of such clocks essentially depends on Newtonian calculations, especially the expected time of arrival $t=d/c$, where $d$ is the distance.  


%Fourth, there is very important observation of A.K.T. Assis that even the so-called wave-particle duality interpretation is untenable in the following sense:


%For indeed Einstein derives $E=mc^2$ from a type of ``kinetic energy" formula for massless light pulses [ref].

%Our response to this third objection is that the above formulas cannot be (and never have been) so used to identify $c^2$ as a velocity. A single formula between three variables, e.g. $E,m,c$ or $u,v,c$, cannot be used to define a variable until the remaining two variables are defined and computable. 

%Our response is that the above quoted formulas are equivalent to Weber Kroschauer formula relating electric and magnetic permittivity.

%So we rebut Rindler's third objection by indicating that apparently ``light" has a definite law, and all fundamental formulas involving light are physically equivalent to (A2) or some weaker form. 


\section{Ralph Sansbury's Experiment}
It appears possible that (A12) is incoherent and contradictory because \textbf{light is not \emph{something} that travels through space}. This provocative idea was introduced to the author by the work of R.Sansbury, for example the very interesting \cite{sansburyspeed} which begins with the following experiment which we quote in full. Recall that $c$ is well approximated at $1$ foot per nanosecond.

\begin{quote}
\emph{ (Case 1) A $15$ nanosecond light pulse from a laser was sent to a light detector, $30$ feet away. When the light pulse was blocked at the photodiode during the time of emission, but unblocked at the expected time of arrival, $31.2$ nanoseconds after the beginning of the time of emission, for 15 nanosecond duration, little light was received. (A little more than the $4mV$ noise on the oscilloscope). This process was repeated thousands of times per second.}

\emph{(Case 2) When the light was unblocked at the photodiode during the time of emission ($15$ nanoseconds) but blocked after the beginning of the time of emission, during the expected time of arrival for $15$ nanoseconds, twice as much light was received ($8mV$). This process was repeated thousands of times per second.}

\emph{This indicated that light is not a moving wave or photon, but rather the cumulative effect of instantaneous forces at a distance. That is, undetectable oscillations of charge can occur in the atomic nuclei of the photodiode that spill over as detectable oscillations of electrons after a delay.}
\end{quote}

This important experiment has apparently not been repeated, despite it's simplicity. We refer the reader to R. Sansbury's book [Ibid] for further details and explanation via \emph{cumulative instantaneous action at a distance}. 

\section{Conclusion}
If "All men are fallible", then logically speaking, any assembly of men and consensus of establishment is also fallible.   This article lays out in plain mathematical terms a persistent error in the first principles of special relativity. The error is subtle, and easily overlooked, yet remains fatal. It is possible that the error stems from a deeper error, that light needs be \emph{something that travels} through space. The experimental impossibility of measuring the one-way speed of light is further evidence that the basic assumptions of special relativity are nonphysical. 

%Thus the possibility of persistent errors in the foundations of special relativity, is that too few persons learn the fundamentals, and revisit them over time, and are able to discern the extremely subtle errors. 


%\section{Conclusion}
%This article attempts to outline inconsistencies in the foundations of Einstein's special relativity, and specifically Einstein's attempted resolution of the apparent incompatibility between (A1) and (A2). Einstein's argument suffers from two defects, as we outlined in Section \ref{li}.













%The possibility of such a fundamental error being unnoticed for more than one hundred years appears very po


%The consequences of the Michelson-Morley's null result









%Thus observation strictly supports only a weaker form (A2w) which might be formulated as follows.

%Indeed special relativity forbids the possibility of having two ``synchronized" clocks at distinct spatial positions; only ``local times" are permitted. 

%It is useful to recall an observation of F.K.T. Assis that material pendulums are even unreliable for nonlocal measurements because their oscillation periods are affected by their lengths and gravitational force [include formula]. N.B. the pendulum demonstrates time dilation without any Lorentz contraction. 
%\begin{itemize}
%\item[(A2w)] 
%The average velocity of every closed light trajectory, where initial and final spatial position are identical, is constant in vacuum.
%\end{itemize}

%Einstein chose to further assume that the velocity is everywhere constant along its trajectory. Admittedly this is a supposition, as he writes: 

%The convention that light travels at constant velocity $c$ during every round-trip means (A2) is replaced by (A2'): 

%And presumably the measurements that any observer $K$ might perform to test (A2) are nonlocal. Specifically, the clock by which the observer might reckon the ``velocity" of a light pulse is required to be stationary, fixed in place at the origin, e.g. Fizeau's experiments [ref], [?]. If we limit ourselves to local measurements, then the critics ought to replace (A2) with the further assumption ``that light also propagates ...". 

%But velocity is ex definitio the ratio of \emph{distance} over \emph{time}. It's possible (in logical sense) that the combination of length contraction and time dilation might yield the same measurement for all observers $K, K'$. However the material properties of both rulers and clocks make these measurements untenable. For a history of this controversy, the reader might investigate the problem of the so-called ``one-way" measurement of the speed of light. See \cite{vid} for an amusing presentation. 




%The above gap in Einstein's argument has been investigated by several critical authors, notably \cite{bryant}, \cite{crothers}. Our goal in this article has been to describe this gap in simple mathematical terms. 

%For Einstein, the purpose of his proof was to demonstrate that his so-called law of the propogation of light was indeed compatible with the principle of restricted relativity. 


% -- however it does not and a definite gap remains. 



%If Einstein's Special Relativity be falsified, where does physics stand today? This author reckons that Weber-Amp\`ere's relational electrodynamics as developed by \cite{assis1}, \cite{aw} can restore the 21st century to the correct path of investigation, and which already contains promising germs of unifying gravitation and strong nuclear forces in electrodynamics.



%For material motions the apparent incompatibilities can be resolved by postulating length contraction and time dilations in the directions of uniform velocity. However for the case of light pulses, we find the gap in Einstein's proof is an essential gap between (i) and (ii) which even Lorentz transformations do not resolve. 



%But what are the consequences of such a finding? According to Einstein, the principles of special relativity begin with Galilean relativity, namely the invariance of the laws of kinematics under affine transformations. It is apparently an experimental fact, that we can all discover, that the Laws of physics are the same in all nonaccelerated reference frames. Einstein calls this the Principle of Restricted Relativity: If $K'$ is a coordinate system moving uniformly (and devoid of rotation) with respect to a coordinate system $K$, then natural phenomena run their course with respect to $K'$ according to exactly the same laws as with respect to $K$. 

%Next Einstein claims  In view of the addition of velocities of particles, there is indeed an apparent incompatibility of the principle and the so-called law. 

%Einstein claims this apparent incompatibility can be resolved by Lorentz transformation law, i.e. by postulating length contraction and time dilation in the direction of uniform velocity. However in the case of the light pulses, the subject turns from material in motion to the transmission of light, which is hypothesized to be something that travels through space.




%[Examples: Organ orthogonal vs parallel to its direction of uniform motion has same pitch] [Examples??]



%I don't recall the ``speed of light being a finite constant" as a law of my youth. It was hardly mentioned as i recall. But i did wonder and stand mystified by the sun's rising and setting, moons, ...


%But there is another postulate, which Einstein adds, namely the purpotedly constant finite velocity of light (en vacuo). He even suggests that this is apparent to school children, which is doubtful. That the transmission of light was not instantaneous had been proposed according to Galileo's rough experiments (lamps at a great distance), and more strongly by Cassini/Roemer and the transits of Io across Jupiter. Thus E adds a second postulate: constant velocity of propogation of light. 

%The solution to the apparent contradiction is, according to Einstein, to modify the idea that ``distance" between two points on a rigid body is independant of the uniform motion of the reference body [ref:E, Ch 11, pp.34]. %Now Einstein clearly indicated the apparent contradiction of these two postulates, by comparing two objects in constant relative motion. For instance, a train in uniform motion relative to a fixed embankment. 

%Einstein claimed to have resolved this apparent contradiction by Lorentz transformations. I.e. by postulating length contraction and time dilation in the direction of uniform velocity, c.f. \cite{einstein2019relativity}[Ch.11, pp.39]. The reader can verify that Einstein does not treat the general case, but restricts himself to a velocity parallel to $x$-axis. Unfortunately a more detailed demonstration was left to future more critical authors, e.g. \cite{crothers}, \cite{bryant}, and nearly one hundred years expired before the gap was noticed.



% leads to the constant velocity in all inertial reference frames. However does the reader sense an inherent tension here? Length contraction is a function of material objects moving near speed of light, but not necessarily involves the contraction of light waves themselves.



%It is unfortunate that E did not give detailed demonstration, because therein he would have been disproven! But a mathematician who does not perform his calculations is like a physicist who does not perform his experiments, lest he be disproven! 

%So let us now begin the formal details: E needs formally demonstrate that the equation $r=ct$ in the $K$ frame is mapped onto $r'=c\tau$ in the $K'$ frame. In particular, as we vary over the points $p$ of the sphere $r=ct$, we need establish that the Lorentz image $p'=Lp$ under the Lorentz transformation $L$ describes a sphere in the $K'$ frame. In otherwords, and here we need be cautious, we need establish that the image radius $r'$ is constant with respect to points $p$ satisfying $r=ct$.

%There is a further error in Einstein: an erroneuous ``factorization" statement of $L=B\circ R$, where $B$ is a \emph{boost} parallel to $x$-axis, and $R$ is a \emph{spatial rotation}. [But the correct factorization is: XX]

%Lorentz group= boost along x-axis, composed with all space rotations.[error?]

%As specified by [ref], Einstein's two conditions are: 

%(I) a spherical equation is satisfied $$(x-x_0)^2+...=R^2$$

%(II) the above equation in $x,y,z,R$ defines a sphere if $R$ is constant as $x,y,z$ vary. 



%Einstein is correct that the image of $R=ct$ under Lorentz group has the form of a ``spherical equation", namely $\xi^2+\eta^2+\zeta^2=c^2 \tau^2$; however the equation represents a sphere(!) only if the ``radius" is constant, i.e. the equation holds and both LHS and RHS are constant. 

%Consider the basic Lorentz boost along the $x$-axis at a velocity $v$. The Lorentz change of coordinates is $$\xi=\gamma (x-vt), \eta=y, \zeta=z, \tau=\gamma(t-vx/c^2), $$ where $\gamma$ is the Lorentz factor. We assume $x,y,z,$ are constrained to the sphere $(x-x_0)^2+..=R^2$.

%Now Einstein is correct that we have $$x^2+y^2+...=c^2 t^2$$ and also $$\xi^2+..=c^2 \tau^2,$$ according to the Lorentz transformation.

%If $t,\tau$ are constant, then both equations represent spheres. However, and this is key point, we observe that $\tau$ is not constant with respect to $x,y,z$. This is made evident by the following computation, which [E] neglects to perform in [ref]: namely $dR'/dx$ is nonzero, even when $x$ is constrained to the sphere $R=ct$. This is key computation which [E] neglects to perform.

%By chain rule, $$\frac{dR'}{dx}=\frac{dR'}{d\tau} \frac{d\tau}{dx}+\frac{dR'}{d\zeta} \frac{d\zeta}{dx}\frac{dR'}{d\nu} \frac{d\nu}{dx}\frac{dR'}{d\xi} \frac{d\xi}{dx},$$ and which according to Lorentz transformation [ref], is equal to $-(v/c)\gamma$. Since this expression is decidedly nonzero, it follows that the radius $R'$ is non constant. 

%Counter arguments: (i) aspects of relativity, such as simultaneity, length contraction, time dilation, are already experimentally valid. 

%(ii) the Lorentz image of the sphere is the view relative an observer in $K$ (and which image is not a sphere, but elliptical), whereas the Lorentz image is ``correct" (i.e. spherical) for a $K'$ observer. 



\printbibliography[title={References}]
\end{document}
