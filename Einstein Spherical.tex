\documentclass[12pt]{amsart}
\usepackage{mathrsfs}
\usepackage{amsmath}
\usepackage{amssymb}
\usepackage{amsfonts}
\usepackage{amsopn}
\usepackage{amsthm}
\usepackage{latexsym}
\usepackage[all]{xy}
\usepackage{enumerate}
\usepackage{geometry}
%\usepackage{biblatex}
%\usepackage{hyperref}
%\usepackage[autostyle]{csquotes}
\usepackage{fancyhdr}
\usepackage{graphicx}
\usepackage{wrapfig}
\usepackage{float}

\usepackage[
    backend=biber,
style=alphabetic,
sorting=nyt,
   % style=authoryear-icomp,
    %sortlocale=de_DE,
    %natbib=true,
    %author=true,
%style=verbose,
%journal=true,
%url=true, 
%    doi=false,
%    eprint=true
]{biblatex}
\addbibresource{biblio.bib}

\usepackage[]{hyperref}
\hypersetup{
    colorlinks=true,
}


\newtheorem*{thm}{Theorem}
\newtheorem*{half}{Halfspace Condition}
\newtheorem{lem}[thm]{Lemma}
\newtheorem{prop}[thm]{Proposition}
\newtheorem*{prob}{Problem}
\newtheorem{cor}[thm]{Corollary}
\newtheorem{question}[thm]{Question}
\newtheorem*{flashbang}{FlashBang Principle}
\newtheorem*{Lefschetz}{Lefschetz theorem}
\newtheorem*{hyp}{Hypothesis}
\theoremstyle{definition}
\newtheorem{dfn}[thm]{Definition}
\newtheorem{exx}[thm]{Example}
\theoremstyle{remark}
\newtheorem{rem}[thm]{Remark}
\newcommand{\fh}{\mathfrak{h}}
\newcommand{\bn}{\mathbf{n}}
\newcommand{\bC}{\mathbb{C}}
\newcommand{\bG}{\mathbb{G}}
\newcommand{\bR}{\mathbb{R}}
\newcommand{\fB}{\mathfrak{B}}
\newcommand{\bZ}{\mathbb{Z}}
\newcommand{\bQ}{\mathbb{Q}}
\newcommand{\bq}{\bar{Q}[t]}
\newcommand{\bb}{\bullet}
\newcommand{\del}{\partial}
\newcommand{\sB}{\mathscr{B}}
\newcommand{\sE}{\mathscr{E}}
\newcommand{\mR}{\bR^\times_{>0}}
\newcommand{\conv}{\overbar{conv}}
\newcommand{\sub}{\del^c \psi^c(x')}
\newcommand{\subb}{\del^c \psi^c(x'')}
\newcommand{\hh}{\hookleftarrow}
\newcommand{\bD}{\mathbb{D}}
\newcommand{\Gm}{\mathbb{G}_m}
\newcommand{\uF}{\underline{F}}
\newcommand{\sC}{\mathscr{C}}
\newcommand{\bX}{\overline{X}^{BS/\bQ}}
\newcommand{\sT}{\mathscr{T}}
\newcommand{\sW}{\mathscr{W}}
\newcommand{\sZ}{\mathscr{Z}}

\begin{document}

\title{On Einstein's Alleged Proof of Invariance of Spherical Light Waves in Special Relativity}
\author{J. H. Martel}
\date{\today}

\maketitle

The subject of this brief note is Einstein's alleged proof of the Lorentz invariance of spherical light waves in his special relativity theory. Our purpose is to describe a positive gap in above mentioned proof, and to further demonstrate the invalidity of its conclusion. The purpose of Einstein's alleged proof is to reconcile the fundamental assumptions of special relativity, namely
\begin{itemize}

\item[(A1)] that the laws of physics are the same in all nonaccelerated reference frames, i.e. if $K'$ is a coordinate system moving uniformly (and devoid of rotation) with respect to a coordinate system $K$, then natural phenomena run their course with respect to $K'$ according to exactly the same laws as with respect to $K$. 

\item[(A2)] that light in vacuum propagates along straight lines with constant velocity $c\approx 300,000$ metres per second.

\end{itemize}

%Assumption (A2) was termed ``the Law of Propogation of Light", and being a \emph{law} was therefore subject to assumption (A1). 

In Einstein's own words \cite[Ch.7, 11]{einstein2019relativity}:

\begin{quote} ``\emph{There is hardly a simpler law in physics than that according to which light is propagated in empty space. Every child at school knows, or believes he knows, that this propagation takes place in straight lines with velocity $c=300, 000 m/sec$.\ldots Who would imagine that this simple law has plunged the conscientiously thoughtful physicist into the greatest intellectual difficulties?"} \end{quote}

Notice the occurence of ``law" in the above quote. Einstein assumes that (A2) \emph{is} a \emph{law}, namely the law of propagation of light. To be simultaneously consistent with (A1) then requires (A2) to be satisfied in every nonaccelerated reference frame. However the conjunction of (A1) and (A2), abbreviated (A12), is contradictory according to classical mechanics and Fizeau's law of addition of velocities, to say the least. However Einstein claims -- and this popularly lauded as among his greatest intellectual achievements -- that this is only an \emph{apparent} incompatibility, and which is resolved by postulating that Lorentz transformations relate the measurements of space- and time-coordinates in the reference frames $K, K'$. But we submit that Einstein's argument is incomplete, and his conclusion incorrect. Einstein's error in the first steps of his theory has been elaborated by other authors, notably \cite{bryant}, \cite{crothers}. The present note arises from the author's own study of the controversy, and his attempt to identify the incompatibility in plain mathematical terms. 

The positive gap has a twofold source. Firstly from Einstein's confusing quadratic expressions like \begin{equation}\label{a}\xi^2+\eta^2+\zeta^2=c^2 \tau^2
\end{equation} with ``the equation of a sphere". Nay, \eqref{a} is a three-dimensional cone in the four variables $\xi, \eta, \zeta, \tau$. Yes, it contains numerous ``spherical" two-dimensional subsets. For instance, the standard geometric sphere $S$ centred at the origin simultaneously satisfies \eqref{a} and the further condition $$\frac{1}{2}d(\xi^2+\eta^2+\zeta^2)=\xi d\xi+\eta d\eta +\zeta d\zeta=0 ~~\text{~on~~} S.$$ In short, the standard geometric sphere requires that $\xi^2+\eta^2+\zeta^2$, and consequently $c^2\tau^2$, be \emph{constant}. This leads us to Einstein's second error, namely that the Lorentz $SO(3,1)$ invariance of the quadratic form $h=x^2+y^2+z^2-c^2t^2$ in no way implies the Lorentz invariance of the forms $h_1=x^2+y^2+z^2$ and $h_2=c^2 t^2.$ These two errors imply Einstein's attempt to resolve (A12) is readily invalidated by basic computations.

%The argument is general and applies to split orthogonal groups of arbitrary rank, e.g. $G=SO(n,1)$,$n=1,2,3,\ldots$. %In the most basic case of $SO(1,1)$, the key observation is that a quadratic expression like $x^2=r^2$ in the $xr$-variables does not represent a ``sphere"; the expression represents a sphere of radius $r$ and only if the additional equation $\del x /\del r =0$ holds for every $x$. But the expression $\del x / \del r$ is not a tensor, i.e. satisfies no invariant 

%\section{}
To repeat, Minkowski's form $h=x^2+y^2+z^2-c^2t^2$ is invariant with respect to Lorentz transformations, i.e. the Lie group $G=SO(h)=SO(3,1)$. Invariance says $\xi^2+\eta^2+\zeta^2-c^2 \tau^2$ is numerically equal to $x^2+y^2+z^2-c^2 t^2$ for every Lorentz transform $(\xi, \eta, \zeta, \tau)=\phi.(x,y,z,t)$. But as we argue below, Minkowski's form is essentially the \emph{only} Lorentz invariant quadratic form on $\bR^4$. Now Einstein's argument concerns the geometry of a wave front generated by a light pulse, and the action of the Lorentz group on subsets of the null cone $N=\{h=0\}$. Obviously $N$ is $G$-invariant and defined by the equation $x^2+y^2+z^2=c^2 t^2$. Let $\sC$ be the vector space of (possibly degenerate) quadratic forms $q$ on $\bR^4$, and let $h\in \sC$ be Minkowski's form. 

%The Lorentz group $G$ acts linearly and irreducibly on $\bR^{3,1}=\bR^4$, so there are no proper $G$-invariant linear subspaces. 

%Now Einstein's argument concerns the geometry of a wave front generated by a light pulse, i.e. the action of the Lorentz group on subsets of the null cone $N=\{h=0\}$. Obviously $N$ is $G$-invariant and satisfies $x^2+y^2+z^2=c^2 t^2$. In formal terms, here is our main observation. Let $\sC$ be the vector space of (possibly degenerate) quadratic forms $q$ on $\bR^4$, and let $h\in \sC$ be Minkowski's form. 


%[Note: strictly speaking, we need prove there does not exist any nonzero quadratics $q$ whose restriction $q|_N$ to the null cone is Lorentz invariant]. 

% the Lorentz group $G$ acts linearly and irreducibly on $\bR^{3,1}=\bR^4$, so there are no proper $G$-invariant linear subspaces;

%\item the null cone $N$ contains no nonempty proper $G$-invariant subsets;

%First we observe that $N$ has no nonempty proper $G$-invariant subsets. In formal terms, if $N'\subset N$ is a nonempty $G$-invariant subset such that $N'.\phi\subset N'$ for every $\phi \in G$, then $N'=N$. Second, and more interesting is the following generalization.




\begin{thm}
If $q\in \sC$ is a quadratic form on $\bR^4$ such that the restriction $q|_N$ is $G$-invariant, then $q$ is proportional to $h$. 
\end{thm} 
\begin{proof}
The $G$-action on $N$ is transitive on nonzero vectors. Therefore if $q|_N$ is $G$-invariant, then $q|_N$ is constant. By continuity it follows that $q|_N$ is identically zero since $q|_N=q(0)=0$. So the zero locus of $q$ contains the zero locus of $h$, i.e. \begin{equation}\label{c} N\subset \{q=0\}.
\end{equation}

Now we use a theorem of J. H. Elton to conclude $q$, $h$ are proportional, c.f. \cite{elton2010indefinite}. We outline his argument. Let $q\otimes_\bR \bC$ and $h\otimes_\bR \bC$ be the complexification of the real quadratic forms $q,h$. Thus $q\otimes_\bR \bC: \bR^4 \otimes_\bR \bC \to \bC$ is a complex-valued quadratic form. Elton's argument depends on establishing the following inclusion \begin{equation}\label{b}
\{h\otimes \bC=0\} \subset \{q\otimes \bC=0\}.
\end{equation}
According to the tensor construction we have $h\otimes \bC (x+iy)=h(x)-h(y) + 2i h(x,y)$. Therefore $h\otimes \bC(x+iy)=0$ if and only if $h(x)=h(y)$ and $h(x,y)=0$, where $h(\cdot, \cdot)$ is the bilinear form canonically defined by $h$. Elton's proof reduces to establishing the implication: if \eqref{c} is satisfied, then $h(x)=h(y)$ and $h(x,y)=0$ implies $q(x)=q(y)$ and $q(x,y)=0$ for all $x,y\in \bR^4$. That $q$ vanishes on the null cone $N$ implies $q$  is indefinite if it is not identically zero.

Once the inclusion \eqref{b} is established, Hilbert's Nullstellensatz \cite{eisenbud2013commutative} implies $q\otimes \bC=\lambda \cdot h\otimes \bC$ for some $\lambda \in \bC$. But then obviously $\lambda\in \bR$ and the theorem follows.
%We have three cases supposing $h\otimes \bC(x+iy)=0$.
%Case 1: if $h(x)=0$, then $h(y)=0$. But then $h(x+y)=h(x)+h(y)+2h(x,y)=0$ according to the assumptions, and it follows that $q\otimes \bC(x+iy)=0$. 
%Case 2: if $h(x)>0$, then rescaling we can choose $h(x)=1$. Since $h$ is indefinite, there exists $z$ such that $h(z)<0$
%Therefore $h\otimes \bC (x+iy)=0$ if and only if $h(x)=h(y)$ and $h(x,y)=0$. But $h(x-y)=h(x)+h(y)-2h(x,y) 
\end{proof}

 %This means there exists only one canonical quadratic form with respect to Lorentz transformations.
That is, the Minkowski form is the unique Lorentz invariant quadratic form (modulo scalars) on $\bR^4$ which vanishes on the null cone. Now we return to the subject at hand, namely Einstein's alleged proof that (A12) are compatible with respect to Lorentz symmetry. For illustration, consider the two-dimensional case in the variables $x,t$. Here we find $h=x^2-c^2 t^2$ is invariant with respect to the group $G=SO(1,1)$ generated by $a_\theta:=\begin{pmatrix} \cosh \theta & \sinh \theta \\
\sinh \theta & \cosh \theta
\end{pmatrix}$, where $\theta\in \bR$. We see $SO(1,1)$ is isomorphic to the split multiplicative torus $\bR^\times _{>0}$ using the logarithm. The unit sphere includes two vectors $\langle 1, 1 \rangle,~~ \langle -1,1\rangle,$ and which are mapped by $a_\theta \in SO(1,1) \simeq \bR^\times_{>0}$ to $$\langle \xi,\tau\rangle=\langle \cosh \theta+\sinh \theta, \cosh \theta+\sinh \theta \rangle,~~ \langle -\cosh \theta+\sinh \theta, \cosh \theta-\sinh \theta \rangle.$$ But evidently $\xi^2 \neq x^2=1$ and $\tau^2 \neq t^2=1$ when $\theta\neq 0$. Thus the quadratic forms $h_1=x^2$ and $h_2=t^2$ are not invariant. Likewise we find the image of the unit sphere $x^2=1$ does not correspond to a sphere in $\xi \tau$ coordinates. %In other words the equation $x^2 = c^2 t^2$ is not the equation of sphere when $x,t$ are variable, and likewise $\xi^2=c^2 \tau^2$ is not the equation of sphere when $\xi,\tau$ are both variable and nonconstant. 
%Thus we find a gap in Einstein's purported proof. %The curious reader may consult Einstein's own words, e.g. \cite[Ch.11, pp.39]{einstein2019relativity} and verify that Einstein does not treat the general case but restricts himself to a velocity parallel to $x$-axis.  

%Moreover the algebraic expression \begin{equation}\label{a}\xi^2+\eta^2+\zeta^2=c^2 \tau^2\end{equation} is presumed to represent the ``equation of a sphere". However, the algebraic expression \eqref{a} of four independant variables $\xi, \eta, \zeta,\tau$ does \emph{not} represent a geometric sphere unless, e.g., $$\del \tau / \del \xi=0, ~ \del \tau / \del \eta=0, \del \tau / \del \zeta=0$$ for every $\xi, \eta, \zeta, \tau$ satisfying \eqref{a}. As the above two-dimensional case illustrates, this is not the case.
Thus we find a positive gap in Einstein's argument, c.f. \cite{bryant}, \cite{crothers}. It appears likely to this author that item (A2) is incorrect for another reason, namely that \emph{light} is not something that \emph{travels through space}. A critical examination of this assumption can be found in R. Sansbury's book \cite{sansburyspeed}. Moreover the incompatibility of (A2) with the hypothesis of ``wave--particle" duality is analyzed by A.K.T. Assis \cite[\S 7.2.4, pp.133]{assis1999relational}. This is subject of future investigations.


%The above gap in Einstein's argument has been investigated by several critical authors, notably \cite{bryant}, \cite{crothers}. Our goal in this article has been to describe this gap in simple mathematical terms. 

%For Einstein, the purpose of his proof was to demonstrate that his so-called law of the propogation of light was indeed compatible with the principle of restricted relativity. 


% -- however it does not and a definite gap remains. 



%If Einstein's Special Relativity be falsified, where does physics stand today? This author reckons that Weber-Amp\`ere's relational electrodynamics as developed by \cite{assis1}, \cite{aw} can restore the 21st century to the correct path of investigation, and which already contains promising germs of unifying gravitation and strong nuclear forces in electrodynamics.



%For material motions the apparent incompatibilities can be resolved by postulating length contraction and time dilations in the directions of uniform velocity. However for the case of light pulses, we find the gap in Einstein's proof is an essential gap between (i) and (ii) which even Lorentz transformations do not resolve. 



%But what are the consequences of such a finding? According to Einstein, the principles of special relativity begin with Galilean relativity, namely the invariance of the laws of kinematics under affine transformations. It is apparently an experimental fact, that we can all discover, that the Laws of physics are the same in all nonaccelerated reference frames. Einstein calls this the Principle of Restricted Relativity: If $K'$ is a coordinate system moving uniformly (and devoid of rotation) with respect to a coordinate system $K$, then natural phenomena run their course with respect to $K'$ according to exactly the same laws as with respect to $K$. 

%Next Einstein claims  In view of the addition of velocities of particles, there is indeed an apparent incompatibility of the principle and the so-called law. 

%Einstein claims this apparent incompatibility can be resolved by Lorentz transformation law, i.e. by postulating length contraction and time dilation in the direction of uniform velocity. However in the case of the light pulses, the subject turns from material in motion to the transmission of light, which is hypothesized to be something that travels through space.




%[Examples: Organ orthogonal vs parallel to its direction of uniform motion has same pitch] [Examples??]



%I don't recall the ``speed of light being a finite constant" as a law of my youth. It was hardly mentioned as i recall. But i did wonder and stand mystified by the sun's rising and setting, moons, ...


%But there is another postulate, which Einstein adds, namely the purpotedly constant finite velocity of light (en vacuo). He even suggests that this is apparent to school children, which is doubtful. That the transmission of light was not instantaneous had been proposed according to Galileo's rough experiments (lamps at a great distance), and more strongly by Cassini/Roemer and the transits of Io across Jupiter. Thus E adds a second postulate: constant velocity of propogation of light. 

%The solution to the apparent contradiction is, according to Einstein, to modify the idea that ``distance" between two points on a rigid body is independant of the uniform motion of the reference body [ref:E, Ch 11, pp.34]. %Now Einstein clearly indicated the apparent contradiction of these two postulates, by comparing two objects in constant relative motion. For instance, a train in uniform motion relative to a fixed embankment. 

%Einstein claimed to have resolved this apparent contradiction by Lorentz transformations. I.e. by postulating length contraction and time dilation in the direction of uniform velocity, c.f. \cite{einstein2019relativity}[Ch.11, pp.39]. The reader can verify that Einstein does not treat the general case, but restricts himself to a velocity parallel to $x$-axis. Unfortunately a more detailed demonstration was left to future more critical authors, e.g. \cite{crothers}, \cite{bryant}, and nearly one hundred years expired before the gap was noticed.



% leads to the constant velocity in all inertial reference frames. However does the reader sense an inherent tension here? Length contraction is a function of material objects moving near speed of light, but not necessarily involves the contraction of light waves themselves.



%It is unfortunate that E did not give detailed demonstration, because therein he would have been disproven! But a mathematician who does not perform his calculations is like a physicist who does not perform his experiments, lest he be disproven! 

%So let us now begin the formal details: E needs formally demonstrate that the equation $r=ct$ in the $K$ frame is mapped onto $r'=c\tau$ in the $K'$ frame. In particular, as we vary over the points $p$ of the sphere $r=ct$, we need establish that the Lorentz image $p'=Lp$ under the Lorentz transformation $L$ describes a sphere in the $K'$ frame. In otherwords, and here we need be cautious, we need establish that the image radius $r'$ is constant with respect to points $p$ satisfying $r=ct$.

%There is a further error in Einstein: an erroneuous ``factorization" statement of $L=B\circ R$, where $B$ is a \emph{boost} parallel to $x$-axis, and $R$ is a \emph{spatial rotation}. [But the correct factorization is: XX]

%Lorentz group= boost along x-axis, composed with all space rotations.[error?]

%As specified by [ref], Einstein's two conditions are: 

%(I) a spherical equation is satisfied $$(x-x_0)^2+...=R^2$$

%(II) the above equation in $x,y,z,R$ defines a sphere if $R$ is constant as $x,y,z$ vary. 



%Einstein is correct that the image of $R=ct$ under Lorentz group has the form of a ``spherical equation", namely $\xi^2+\eta^2+\zeta^2=c^2 \tau^2$; however the equation represents a sphere(!) only if the ``radius" is constant, i.e. the equation holds and both LHS and RHS are constant. 

%Consider the basic Lorentz boost along the $x$-axis at a velocity $v$. The Lorentz change of coordinates is $$\xi=\gamma (x-vt), \eta=y, \zeta=z, \tau=\gamma(t-vx/c^2), $$ where $\gamma$ is the Lorentz factor. We assume $x,y,z,$ are constrained to the sphere $(x-x_0)^2+..=R^2$.

%Now Einstein is correct that we have $$x^2+y^2+...=c^2 t^2$$ and also $$\xi^2+..=c^2 \tau^2,$$ according to the Lorentz transformation.

%If $t,\tau$ are constant, then both equations represent spheres. However, and this is key point, we observe that $\tau$ is not constant with respect to $x,y,z$. This is made evident by the following computation, which [E] neglects to perform in [ref]: namely $dR'/dx$ is nonzero, even when $x$ is constrained to the sphere $R=ct$. This is key computation which [E] neglects to perform.

%By chain rule, $$\frac{dR'}{dx}=\frac{dR'}{d\tau} \frac{d\tau}{dx}+\frac{dR'}{d\zeta} \frac{d\zeta}{dx}\frac{dR'}{d\nu} \frac{d\nu}{dx}\frac{dR'}{d\xi} \frac{d\xi}{dx},$$ and which according to Lorentz transformation [ref], is equal to $-(v/c)\gamma$. Since this expression is decidedly nonzero, it follows that the radius $R'$ is non constant. 

%Counter arguments: (i) aspects of relativity, such as simultaneity, length contraction, time dilation, are already experimentally valid. 

%(ii) the Lorentz image of the sphere is the view relative an observer in $K$ (and which image is not a sphere, but elliptical), whereas the Lorentz image is ``correct" (i.e. spherical) for a $K'$ observer. 



\printbibliography[title={References}]
\end{document}
