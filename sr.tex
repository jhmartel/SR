\documentclass[12pt]{amsart}
\usepackage{mathrsfs}
\usepackage{amsmath}
\usepackage{amssymb}
\usepackage{amsfonts}
\usepackage{amsopn}
\usepackage{amsthm}
\usepackage{latexsym}
\usepackage[all]{xy}
\usepackage{enumerate}
\usepackage{geometry}
%\usepackage{biblatex}
%\usepackage{hyperref}
%\usepackage[autostyle]{csquotes}
\usepackage{fancyhdr}
\usepackage{graphicx}
\usepackage{wrapfig}
\usepackage{float}

\usepackage[
    backend=biber,
style=alphabetic,
sorting=nyt,
   % style=authoryear-icomp,
    %sortlocale=de_DE,
    %natbib=true,
    %author=true,
%style=verbose,
%journal=true,
%url=true, 
%    doi=false,
%    eprint=true
]{biblatex}
\addbibresource{biblio.bib}

\usepackage[]{hyperref}
\hypersetup{
    colorlinks=true,
}


\newtheorem*{thm}{Theorem}
%\newtheorem*{half}{Halfspace Condition}
\newtheorem*{lem}{Lemma}
%\newtheorem{prop}[thm]{Proposition}
%\newtheorem*{prob}{Problem}
%\newtheorem{cor}[thm]{Corollary}
%\newtheorem{question}[thm]{Question}
%\newtheorem*{flashbang}{FlashBang Principle}
%\newtheorem*{Lefschetz}{Lefschetz theorem}
%\newtheorem*{hyp}{Hypothesis}
%\theoremstyle{definition}
%\newtheorem{dfn}[thm]{Definition}
%\newtheorem{exx}[thm]{Example}
%\theoremstyle{remark}
%\newtheorem{rem}[thm]{Remark}
\newcommand{\fh}{\mathfrak{h}}
\newcommand{\bn}{\mathbf{n}}
\newcommand{\bC}{\mathbb{C}}
\newcommand{\bG}{\mathbb{G}}
\newcommand{\bR}{\mathbb{R}}
\newcommand{\fB}{\mathfrak{B}}
\newcommand{\bZ}{\mathbb{Z}}
\newcommand{\bQ}{\mathbb{Q}}
\newcommand{\bH}{\mathbb{H}}
\newcommand{\bq}{\bar{Q}[t]}
\newcommand{\bb}{\bullet}
\newcommand{\del}{\partial}
\newcommand{\sB}{\mathscr{B}}
\newcommand{\sE}{\mathscr{E}}
\newcommand{\mR}{\bR^\times_{>0}}
\newcommand{\conv}{\overbar{conv}}
\newcommand{\sub}{\del^c \psi^c(x')}
\newcommand{\subb}{\del^c \psi^c(x'')}
\newcommand{\hh}{\hookleftarrow}
\newcommand{\bD}{\mathbb{D}}
\newcommand{\Gm}{\mathbb{G}_m}
\newcommand{\uF}{\underline{F}}
\newcommand{\sC}{\mathscr{C}}
\newcommand{\bX}{\overline{X}^{BS/\bQ}}
\newcommand{\sT}{\mathscr{T}}
\newcommand{\sW}{\mathscr{W}}
\newcommand{\sZ}{\mathscr{Z}}

\begin{document}


\title[Foundations of SR and Light Propagation in Vacuum]{On Critical Foundations of Special Relativity and Light Propagation in Vacuum}
%Lorentz Non Invariance of Spherical Lightwaves in Special Relativity}


\author{J. H. Martel}
\date{\today}
\email{jhmartel@proton.me}
\maketitle

\begin{abstract}
%This note examines whether spherical light waves are Lorentz invariant or non invariant in Einstein's special relativity. This note argues that the spatial quantity called ``radius" in the literature is not a Lorentz invariant quantity. Equivalently we argue that the homogeneous wave equation does not have a nontrivial class of Lorentz invariant radial solutions. Thus we find the hypothesis that wave fronts generated by light pulses are round spheres in every inertial frame is not consistent with the principle of relativity. This leads to a reexamination of the assumptions underlying Einstein's special relativity theory. This controversial subject has already been developed by \cite{bryant}, \cite{crothers}. The present article aims to present the controversy in elementary mathematical terms.
\end{abstract}
\tableofcontents







\section{Introduction}
This article is a critical essay on the mathematical foundations of special relativity (SR) and the propagation of light in vacuum. The present article arose from the author reviewing several articles \cite{bryant}, \cite{crothers} which claimed there were subtle errors in Einstein's original proof of the Lorentz invariance of spherical light waves, see \S \ref{einsteinproof} below. The purpose of this article is to present some controversial issues in SR in plain mathematical terms. Thus we hope this article is interesting to both adherents and skeptics of the SR theory. For something of the extremely interesting history of skepticism and Einstein, the reader might be interested in \cite{kennefick}.

%, but also his assistant N. Rosen and H. Bondi at various points. Our purpose is not to invalidate the entire field of SR, and this is even impossible since most results are correct formal consequences of their assumptions.

% But rather our goal is pedagogical, seeking to simply elaborate important and interesting issues regarding the hypotheses of SR and the propagation of light in vacuum.



%We emphasize the problem of propagation of light in vacuum, leaving the interpretation of SR to matter aside.




%Controversy and skepticism has always been an important aspect of Einstein's revolution of 20th century physics (c.f. \cite{kennefick}). The author's interest in this subject began by reviewing \cite{bryant}, \cite{crothers}, which claimed there were errors in Einstein's original proof of the Lorentz invariance of \emph{spherical light waves}. Our review of these errors is discussed in \S \ref{einsteinproof} below. This error was correctly identified by the previous authors, but does not -- in this author's opinion -- necessarily disrupt the development of SR. Instead the error indicates that SR in practice requires additional assumptions concerning the propagation of light in vacuum.


%A careful review of modern approaches to SR, especially Penrose-Rindler's textbook \cite{penrose1984spinors} correctly identifies that the null sphere has a Lorentz invariant conformal structure, but not a Riemannian structure. In this 



The sections of this essay are structured as follows. In \S \ref{s1} we present Einstein's assumptions (A1), (A2) which logically underpin SR, and the admitted apparent incompatibility in their conjunction (A12). In \S \ref{s3} we introduce the Lorentz formulae and the null cone and consider consequences of (A12) in the photon model of light, but here find the law of propagation of the photon particle is \emph{underdetermined} by (A12). This is counter to the Riemannian intuition, and says one cannot determine the time-position or motion of a photon moving along a straight line with speed $c$ in vacuum. Indeed the photon's motion with time still has degrees of freedom even though bound to a line and having speed $c$. In \S \ref{einsteinproof} we examine Einstein's proof of the compatibility of (A12) and his appeal to spherical light waves. We here identify two critical errors in Einstein's argument, but summarized in the fact that $r$ is not a Lorentz invariant variable. In \S \ref{compute} we illustrate with some elementary computations in the $1+1$ spacetime $\bR^{1,1}$. Thus we argue that (A2) is an incomplete description of light, and additional assumptions are necessary. These further assumptions were provided by Levi-Civita in the formulas of geometric optics. The wave model is considered in \S \ref{wes} where we examine the homogeneous wave equation (HWE). The Lorentz non invariance of spherical lightwaves is phrased in terms of the impossibility of defining a Lorentz invariant \emph{radius} and a Lorentz invariant class of radial solutions to HWE. In \S \ref{objections} we try to anticipate and address some counter-objections to our arguments. In \S \ref{sansbury} we enter into controversial issue of Ralph Sansbury's experiment, and an intriguing alternative to the assumptions (A2). 




% namely that light trajectories are somehow \emph{limits} of material trajectories and satisfy a type of geodesic equation $\nabla_{\gamma'} \gamma'=0$. In \S \ref{optics} we discuss the theoretical hazard associated with ``geodesics of zero length" and curves on the null cone. We warn against careless applications of Riemannian reasoning to the Lorentzian null cones. Specifically we argue that the variational equation $\delta \int ds=0$ reduces to the trivial identity $0=0$ throughout the null cone $N=\{ds=0\}$, and is \emph{not} equivalent to the geodesic-type equation $\nabla_{\gamma'} \gamma'=0$ (e.g. $x_i''=0$) like in Riemannian geometry $(ds>0)$, where $\nabla$ is the covariant derivative. The standard proofs of this convergence require $ds$ to be an independant variable, but again $ds$ is identically zero on $N$. This is elaborated in more detail below.



%But what is the relation of HWE to SR? It is true that Maxwell's equations are independant of SR and (A12), although they are of course closely connected. Evidently Einstein himself was motivated by Maxwell's equations concerning the $E,B$ fields, and his attempt for a principle of relativity to hold for the fields $E,B$ and $E', B'$ of two observers in frames $K, K'$. But it appears that a natural duality between quadratic symmetric line elements $$ds^2=-c^2 dt^2 + dx^2+dy^2+dz^2$$ and the second order differential operator $$\square=\square_{1,3}=-\frac{1}{c^2} \frac{\del^2}{\del dt^2} + \frac{\del^2}{\del dx^2}+\frac{\del^2}{\del dy^2} +\frac{\del^2}{\del dz^2}.$$ Thus curves in the null cone $ds=0$ are dual to solutions of $\square \phi=0$ for $\phi \in C(\bR^{1,3})$. 








\section{Assumptions and Difficulties in SR}\label{s1}
We begin our study of the foundations of SR with Einstein's presentation of the theory \cite{einstein2019relativity}, \cite{einstein1905electrodynamics}. In Einstein's words \cite[Ch.7, 11]{einstein2019relativity}:

\begin{quote} 
``\emph{There is hardly a simpler law in physics than that according to which light is propagated in empty space. Every child at school knows, or believes he knows, that this propagation takes place in straight lines with velocity $c=300,000 ~km/sec$ [one foot per nanosecond].\ldots Who would imagine that this simple law has plunged the conscientiously thoughtful physicist into the greatest intellectual difficulties?"} 
\end{quote}

The elaboration of the ``intellectual difficulties" of this supposedly ``simple law" is the main subject of this essay. The primary difficulty is to reconcile the following axiomatic assumptions of SR, namely:
\begin{itemize}

\item[(A1)] that the laws of physics are the same in all nonaccelerated reference frames, i.e. if $K'$ is a coordinate system moving uniformly (and devoid of rotation) with respect to a coordinate system $K$, then natural phenomena run their course with respect to $K'$ according to exactly the same laws as with respect to $K$. 

\item[(A2)] that light in vacuum propagates along straight lines with constant velocity $c\approx 300,000 ~km/sec$ [one foot per nanosecond].

\end{itemize}

Assumption (A1) is generally known as the \emph{principle of restricted relativity}, and represents a meta principle that the \emph{form} of the laws of physics need be the same for all observers in uniform relative motion. This assumption introduces the class of inertial frames and raises the question of how to ``change variables" between different inertial frames $K, K'$.

It's left somewhat undefined what a ``law" is, but we might assume law includes mathematical laws. More specifically (A2) is assumed to represent such a law, and this is clear from the ``schoolboy" quoted above. The assumption (A2) is coined the \emph{law} of \emph{propagation} of light in vacuum. 

The great intellectual difficulty arises from the conjunction of (A1) and (A2), abbreviated (A12). If (A2) is to be consistent with (A1), then the law (A2) must hold true in every inertial reference frame $K'$. But (A12) appears contradictory to classical mechanics, e.g. Fizeau's law of addition of velocities and the additivity of energy. Yet Einstein claims the difficulty is only an \emph{apparent incompatibility}. The incompatibility is reconciled according to Einstein's reasoning by postulating Lorentz-Fitzgerald's length contractions and time dilations, c.f. \cite[Ch.XIV]{michelson}. This claim that Lorentz transformations reconcile (A12) in fact contains two assertions: 

\begin{itemize}
\item[(i)] that inertial frames $K, K'$ are related by Lorentz transformations; 

\item[(ii)] that the law of propagation of light (A2) is Lorentz covariant, i.e. if (A2) is satisfied in $K$, then (A2) is satisfied in every Lorentz translate $K'=\lambda.K$.
\end{itemize}

The assumption (A2) does not explicitly assume either a particle or wave model of light. As we describe below, Einstein attempts to prove assertion (ii) from a wave-theoretic viewpoint, arguing in terms of light pulses and spherical wavefronts. However the Minkowski linear algebra which arises from the Lorentz transformations represents the photon corpuscular model. Our critical analysis applies to both models, and in fact we argue that ultimately (A12) is incompatible with \emph{both} the particle and wave interpretations of light. We develop our argument and line of reasoning in the following sections.  




%Thus what Einstein needs demonstrate is that the formulae of the Lorentz transformations preserves the form of the law (A2) in every inertial frame. In the next sections we will review Einstein's proof and his interpretation of spherical light waves. This leads us to examine the action of Lorentz transformations on the law (A2) and the relation to luminal spherical waves in the next sections. 





%[include later] But Einstein's picture of spherical light waves is somewhat peculiar, especially since Einstein (1905) argued against Maxwell's field theory, and proposed the photon particle concept as an explanation of the photoelectric effect. So for Einstein to argue in terms of the spherical light wave is suprising. If we look to (A2), then we find that it is an apparently well-formed sentence, except on examination, it's perhaps too Riemannian and relates to light propagation \emph{as if} it's something measurable and definable for the Riemannian geometer. 


\section{Photons and Null Cone}\label{s3}

%\subsection{Lorentz Transformations and Minkowski's Form $h=ds^2$}\label{li}

For a proper analysis of (A2) we must remind the reader of the distinction between ``velocity" and ``speed".  As is well known, velocity is a vector in a tangent space, an ``arrow" which has a magnitude (``speed") and a direction. We read (A2) as declaring that light propagates in vacuum in fixed directions (along straight lines) and with constant magnitude (the ``speed" $c$). This is the standard Riemannian conception of vector. Yet in Lorentzian geometry, the magnitude or ``speed" loses its Riemannian meaning, and as we argue below, the law of propagation (A2) is strongly underdetermined from the Riemannian perspective. 





%Comment on Michelson Morley: it leaves undetermined whether the observed null effect is due to clocks accelerating in the transverse direction, or whether they are slowing down in the tangent direction. by convention, einstein and the lorentz formula supposed that the clocks in the tangent direction, where they expected the time delay to be shorter must have experienced a contraction. In Lorentz there is both a length contraction, and a slow down of the local time (relative to who?).

The linear algebra of Minkowski and Lorentz transformations plays a definitive role in SR. The linear group of Lorentz transformations was hypothesized as an attempt to explain the observed null result of Michelson-Morley's interferometer experiment. The experiment was intended to measure variations of the speed of light relative to the \emph{aether}. No such variations were discovered, and it was \emph{postulated} that the usual space and time coordinates $(x,y,z,t)$ and $(\xi, \eta, \zeta, \tau)$ of two inertial observers $K$ and $K'$, respectively, were not related by Galilean transformations, but related by the Lorentz-Fitzgerald formulae. The heuristic was that incredibly and contrary to all expectations, the material arm of the interferometer contracted in the direction of motion and simultaneously the time parameter was contracted by the same ratio, namely the so-called gamma factor $\gamma = 1/\sqrt{1-v^2/c^2}$. A review of the history of mechanics \cite{dugas} shows that Voigt studied a similar set of transformations of solutions to the homogeneous wave equations, but which solutions did not form a group under composition. Lorentz transformations in the setting of SR can be defined as the group of linear transformations $\lambda: \bR^{4} \to \bR^4$ which satisfy $\lambda^*(h)=h$ where $h=ds^2$ is the Minkowski-Lorentz quadratic form \begin{equation}\label{mform}
h:=ds^2=-c^2dt^2+dx^2+dy^2+dz^2.\end{equation} 
Here $c$ is the constant luminal velocity in vacuum posited by (A2). The Lorentz invariance of $h$ says $\xi^2+\eta^2+\zeta^2-c^2 \tau^2$ is numerically equal to $x^2+y^2+z^2-c^2 t^2$ for every Lorentz transform $\lambda$ satisfying $(\xi, \eta, \zeta, \tau)=\lambda \cdot (x,y,z,t)$. It is assumed that $h$ is a scalar invariant for all inertial observers. We remark that Minkowski's form is the \emph{only} Lorentz invariant quadratic form on $\bR^4$ modulo homothety, c.f. \cite{elton2010indefinite}, \cite{arminjon2018lorentz}. 

% need rewrite below.
%The same calculation which proves the Lorentz invariance of the Minkowski form $ds^2$ also seems to imply the Lorentz invariance of the algebraic ``dual" to $ds^2$ which is the $(3+1)$ homogeneous wave equation (HWE), see \eqref{hw}. Thus from the linear algebra perspective there is correspondance between the null cone $ds=0$ and solutions to the HWE $\square \phi =0$. We elaborate below.



%This merely formalizes the observational fact that the proof of Lorentz invariance are basically equivalent.
%As is well known, the choice of a quadratic bilinear form, here $ds^2$ allows an identification between differential forms and vector fields (i.e. derivations). T



%can be seen in the Maxwell field equations in a region with zero electric charge $\rho=0$ and zero current density $J=0$. Composing the field equations $$\nabla \cdot E =0, ~~~\nabla \cdot B =0 $$ and $$\nabla \times E = -\frac{\del B}{\del t}, ~~ \nabla \times B = \mu_0 \epsilon_0 \frac{\del E}{\del t},$$ we apply an algebraic identity for ``curl of the curl" and find both $E, B$ satisfy homogeneous wave equations with $c^2=\mu_0 \epsilon_0$. Thus Maxwell's equations describe how variations in the electromagnetic field propagate at speed $c$ in the aether medium. This is how Maxwell derived the idea that light is a form of electromagnetic radiation. 


%% Maxwell shows that the electric field propagates like a wave at speed $c$. But what is light and the electric field??






%We caution the reader against casually setting $c:=1$ and treating $t$ as a space variable immediately comparable to $x,y,z$. The constant $c$, whether its numerical value is $1$ or not (and with respect to which units) is necessary to transition from time units to space units. 


Now we say something about the photon model of light as treated in \cite[III.XI.6, pp.301]{levi}. If light satisfies (A2), then in a reference frame $K$ light is \emph{something} $\gamma(t)=(x(t),y(t),z(t))$ that travels through space with time, and whose velocity $\gamma'$ \emph{if it could be materially measured as a function of $t$} would satisfy \begin{equation}\label{vel} 
||\gamma'||^2=(dx/dt)^2+(dy/dt)^2+(dz/dt)^2=c^2.
\end{equation} Thus it is argued that light trajectories are constrained to the null cone $$N:=\{h=0\}$$ of Minkowski's metric $h$. Obviously the null cone $N$ is Lorentz invariant subspace of ${\bf{R}}^{1,3}$. and defined by the equation $$x^2+y^2+z^2=c^2 t^2$$ in any reference frame $K$ with coordinates $(x,y,z,t)$.   

%The hypothesis that light is something that can be coordinatized in $x,y,z$ coordinates 

%Remark. The expression $\gamma=\gamma(t)$ is deceivingly simple, since it's not clear that light photons can be represented as corpuscles with a parameterization $\gamma(t)$ depending on time $t$. And here $t$ is taken to be the canonical ``time" coordinate relative to $K$. 


What does the identity \eqref{vel} say about the propagation of light? Typically it's argued that \eqref{vel} demonstrates the \emph{speed} of $\gamma$ is identically equal to $c$ as measured by $K$. The Lorentz invariance of the cone $N$ implies that differentiable curves $\gamma(t)$ contained in $N$ will represent light trajectories exhibiting a constant speed $c$ \emph{independant of the trajectory of $\gamma$ on $N$}. Einstein's (A2) postulates that the propagation is along straight lines, i.e. with constant \emph{direction} in vacuum. This straight line propagation is a convenient hypothesis since it's preserved by the linearity of the Lorentz action. Thus the hypothesis (A2) that light propagates along straight lines with constant \emph{speed} $c$ in every inertial reference frame $K'$ \emph{is} invariant with respect to Lorentz transformations. Thus we admit that the conjunction (A12) is consistent up to this point. 

However the above facts notwithstanding, we maintain that the propagation of light remains \emph{underdetermined} by (A2) from the point of view of Riemannian geometry. Indeed in Riemannian geometry, if a particle is travelling in a straight line and with constant velocity, then the propagation of the particle, namely it's position as a function of time, \emph{is} uniquely determined. However, in Lorentzian geometry, a particle which is travelling in a straight line along the null cone will always have a constant speed, regardless of its trajectory. Supposing that the trajectory is confined to a straight line, there still remains the question of the position of the particle as a function of time. The problem is that the \emph{uniformity} of the straight line propagation is meaningless on the null cone of Lorentzian geometry. 

Here we introduce Levi-Civita's approach as represented in his excellent text \cite{levi}. Levi-Civita modifies (A2) somewhat by asserting that ``the propagation of light is rectilinear, uniform, and with velocity $c$". The term ``uniform" does not feature in Einstein's formulation of (A2), although it speaks to the hidden assumption that the light rays have a canonical parameter (describing the \emph{uniform} motion of the light ray). The above remarks are directly related to Levi-Civita's characterization of geometric optics in the following two equations (see \cite[III.XI.16]{levi}): \begin{equation}
\delta \int ds=0
\label{var}\end{equation}
and
\begin{equation}
ds^2=0.
\end{equation}

The first equation \eqref{var} says the variational derivative of the functional $\gamma \mapsto \int_\gamma ds$ vanishes on the light trajectories, and the second equation says the trajectory is constrained to the null cone. In the Riemannian setting where $ds$ is positive definite, the equation \eqref{var} is essentially equivalent to the geodesic equation $\nabla_{\gamma'} \gamma'=0$. However in the Lorentzian setting we find \eqref{var} reduces to $0=0$ on the null cone $N$. Thus the usual Riemannian $ds>0$ argument does \emph{not} establish the corresponding ``geodesic" equation on $N$. This is acknowledged in \cite[III.XI.14]{levi} but Levi-Civita argues that zero length geodesics are limits of Riemannian geodesics ($ds>0$) and that ``there is a process of passing to the limit (in conditions of complete analytical regularity) from ordinary geodesics". Levi-Civita maintains that the variational equation \eqref{var} somehow ``implies" the geodesic-type equation $\nabla_{\gamma'} \gamma'=0$ for light rays, c.f. \cite[III.XI.18]{levi}. 

Our viewpoint is that $\nabla_{\gamma'} \gamma'=0$ is an \emph{independant} hypothesis, and by no means a formal consequence of \eqref{var}. This is related to our contention that contrary to Levi-Civita's claims, the variational equation $$\delta \int ds =0$$ on $N=\{ds=0\}$. In Riemannian geometric terms,  we find straight lines on $N$ have \emph{no} canonical parameterizations, even affine. This reveals a clear distinction between Riemannian straight lines which \emph{do} have a canonical arclength parameter $ds$, and the null lines $\ell \subset N$ which do \emph{not} admit canonical $ds$ arclength parameter \emph{except the trivial $ds=0$}. 


%So what does ``uniform" in (A2) \emph{really} say about the propagation of light? We argue that (A2) appears deterministic, as if it's specifying the propagation of light. But in fact we reason that (A2) is a weak law, that does not really specify \emph{how} or \emph{when} or \emph{where} light is propagated.

%For example, we commented earlier on the relation of the HWE as linear algebraically dual to $ds^2$. Now (A12) asserts that light trajectories are constrained to the null cone $ds=0$, but what is the \emph{dual} interpretation of (A12) in terms of the homogeneous wave equation? 

%Thus there appears an implicit assumption in (A2), namely that light propagates \emph{uniformly} according to an affine function of $t$. 

%This hidden assumption becomes more prominent in the wave or undulatory model of light. 
%What does this mean physically? 
%Apples and oranges. Cannot drop the constant c and treat x,t as comparable quantities.

%The inertial observers $K, K', \ldots$ of $\bR^3$ are those reference frames, where any pair $K, K'$ of inertial observers are assumed to have a constant relative velocity. This defines the set of inertial frames in bijection with the group of affine motions $\text{Aff}(\bR^3)$ of $\bR^3$. (Formally, the inertial frames are an example of a principal $\text{Aff}(\bR^3)$-space). [Ref] 




\section{Critique of Einstein's Proof of Compatibility of (A12)}\label{einsteinproof}
Now we turn to our critical analysis of Einstein's ``proof" of the compatibility of (A12). We were much influenced by the main results of \cite{bryant}, \cite{crothers}. 

Einstein's proof looks to derive the assertion (ii) from \S \ref{s1} as a consequence of assertion (i). We claim that the errors in Einstein's attempted proof are twofold. \textbf{First an error arises when quadratic expressions like \begin{equation}\label{a}\xi^2+\eta^2+\zeta^2=c^2 \tau^2
\end{equation} are \emph{misidentified} with ``the equation of a sphere".} Strictly speaking \eqref{a} is a three-dimensional cone in the four independant variables $\xi, \eta, \zeta, \tau$. Of course the cone contains many spherical two-dimensional subsets, but our point is that a second independant equation is necessary to specify these metric spheres. 

The Penrose approach is to projectivize the equation \eqref{a} and obtain a projective sphere with a well defined conformal structure \cite[Ch 1.]{penrose1984spinors}. But again there is no canonical metric invariant with respect to the Lorentz group on the projectivization. In otherwords, the null sphere has a Lorentz invariant \emph{conformal} structure, but it does not have a Lorentz invariant \emph{Riemannian metric} structure. For instance the standard round sphere $S$ centred at the origin simultaneously satisfies \eqref{a} and additionally the equation $$\frac{1}{2}d(\xi^2+\eta^2+\zeta^2)=\xi d\xi+\eta d\eta +\zeta d\zeta=0.$$ In short, the round sphere requires that \emph{two} quadratic forms $\xi^2+\eta^2+\zeta^2$ and $c^2\tau^2$ be \emph{simultaneously constant}. 

This leads us to Einstein's second error, which is the failure to observe that \textbf{the Lorentz invariance of the quadratic form $h=x^2+y^2+z^2-c^2t^2$ in no way implies the Lorentz invariance of $h_1:=x^2+y^2+z^2$ and $h_2:=c^2 t^2.$ }. Indeed the quadratic forms $h_1, h_2$ are degenerate, with nontrivial radicals $rad(h_1)=\{x=y=z=0\}$ and $rad(h_2)=\{t=0\}.$ The radicals are linear subspaces of $\bR^4$. But if $h_1, h_2$ are invariant, then $rad(h_1)$ and $rad(h_2)$ are also nontrivial invariant subspaces. This contradicts the fact that the standard linear representation of the Lorentz group acts irreducibly on $\bR^4$.




%The argument is general and applies to split orthogonal groups of arbitrary rank, e.g. $G=SO(n,1)$,$n=1,2,3,\ldots$. %In the most basic case of $SO(1,1)$, the key observation is that a quadratic expression like $x^2=r^2$ in the $xr$-variables does not represent a ``sphere"; the expression represents a sphere of radius $r$ and only if the additional equation $\del x /\del r =0$ holds for every $x$. But the expression $\del x / \del r$ is not a tensor, i.e. satisfies no invariant 

%\section{}



%Now returning to (A12), we look with Einstein to the wave fronts generated by light pulses. 







%We include the following result, essentially due to J.H. Elton. For notation, let $\sC$ be the vector space of possibly degenerate quadratic forms $q$ on $\bR^4$, and let $h\in \sC$ be Minkowski's form. 

%\begin{thm}
%If $q\in \sC$ is a quadratic form on $\bR^4$ such that the restriction $q|_N$ is $G$-invariant, then $q$ is proportional to $h$. 
%\end{thm} 
%\begin{proof}
%The $G$-action on $N$ is transitive on nonzero vectors. Therefore if $q|_N$ is $G$-invariant, then $q|_N$ is constant. By continuity it follows that $q|_N$ is identically zero since $q(0)=0$. So the zero locus of $q$ contains the zero locus of $h$, i.e. \begin{equation}\label{c} N\subset \{q=0\}.
%\end{equation}

%Now we use a theorem of J. H. Elton to conclude $q$, $h$ are proportional, c.f. \cite{elton2010indefinite}. We outline his argument. Let $q\otimes_\bR \bC$ and $h\otimes_\bR \bC$ be the complexifications of the real quadratic forms $q,h$. Thus $q\otimes_\bR \bC: \bR^4 \otimes_\bR \bC \to \bC$ is a complex-valued quadratic form. Elton's proof establishes the following inclusion \begin{equation}\label{b}
%\{h\otimes \bC=0\} \subset \{q\otimes \bC=0\}.
%\end{equation}
%According to the tensor construction we have $h\otimes \bC (x+iy)=h(x)-h(y) + 2i h(x,y)$. Therefore $h\otimes \bC(x+iy)=0$ if and only if $h(x)=h(y)$ and $h(x,y)=0$, where $h(\cdot, \cdot)$ is the bilinear form canonically defined by $h$. Elton's proof reduces to establishing the implication: if \eqref{c} is satisfied, then $h(x)=h(y)$ and $h(x,y)=0$ implies $q(x)=q(y)$ and $q(x,y)=0$ for all $x,y\in \bR^4$. That $q$ vanishes on the null cone $N$ implies $q$  is indefinite if it is not identically zero. Once the inclusion \eqref{b} is established, Hilbert's Nullstellensatz \cite{eisenbud2013commutative} implies $q\otimes \bC=\lambda \cdot h\otimes \bC$ for some $\lambda \in \bC$. But then obviously $\lambda\in \bR$ and the theorem follows.
%We have three cases supposing $h\otimes \bC(x+iy)=0$.
%Case 1: if $h(x)=0$, then $h(y)=0$. But then $h(x+y)=h(x)+h(y)+2h(x,y)=0$ according to the assumptions, and it follows that $q\otimes \bC(x+iy)=0$. 
%Case 2: if $h(x)>0$, then rescaling we can choose $h(x)=1$. Since $h$ is indefinite, there exists $z$ such that $h(z)<0$
%Therefore $h\otimes \bC (x+iy)=0$ if and only if $h(x)=h(y)$ and $h(x,y)=0$. But $h(x-y)=h(x)+h(y)-2h(x,y) 
%\end{proof}

 %This means there exists only one canonical quadratic form with respect to Lorentz transformations.
%That is, the Minkowski form is the unique Lorentz invariant quadratic form (modulo scalars) on $\bR^4$ which vanishes on the null cone. Another argument is possible \cite{arminjon2018lorentz} which establishes nonexistence of any other Lorentz invariant $(0,2)$-tensors (the proof is long algebraic computation). Naturally the question arises whether there exists any other invariant $(p,q)$-tensors which are independant of Minkowski's form $h$.  

\section{Elementary Computations in ${\bR}^{1,1}$}\label{compute}

Now we present a simple computation to illustrate the numerics involved in (A12), and to illustrate the basic ideas of the previous sections. We restrict ourselves to two variables $(x,t)$ and $(\xi, \tau)$. For numerical convenience we set $c:=1$. Thus $h=x^2-t^2$ is a quadratic form on $\bR^2$ invariant with respect to the one-dimensional Lorentz group $G=SO(1,1)_0$ generated by $$a_\theta:=\begin{pmatrix} \cosh \theta & \sinh \theta \\
\sinh \theta & \cosh \theta
\end{pmatrix}$$ for $\theta\in \bR$. In two dimensions the null cone $$N=\{x^2-t^2=0\}$$ projectivizes to a $0$-dimensional sphere consisting of two projective points represented by the affine lines $x-t=0$ and $x+t=0$. The round $0$-dimensional sphere $\{x^2=1\}$ consists of two vectors in the null cone, namely $\begin{pmatrix} 1 \\ 1\end{pmatrix}$ and $\begin{pmatrix} -1 \\ 1\end{pmatrix}.$ Left translating these vectors by $a_\theta$ we find the translates $$\begin{pmatrix} \xi \\ \tau \end{pmatrix}=\begin{pmatrix} \cosh \theta+\sinh \theta \\ \cosh \theta+\sinh \theta \end{pmatrix}~~\text{~and~} \begin{pmatrix} -\cosh \theta+\sinh \theta \\ \cosh \theta-\sinh \theta \end{pmatrix}.$$ But evidently $$\xi^2 \neq x^2=1^2=1 ~~\text{~and~}~~ \tau^2 \neq t^2=1$$ when $\theta\neq 0$. Thus the quadratic forms $h_1=x^2$ and $h_2=t^2$ are not $a_\theta$-invariant. Likewise we find the image of the unit sphere $x^2=1$ does not correspond to a spatial sphere in $(\xi, \tau)$ coordinates. 

These trivial computations have the effect of falsifying the alleged Lorentz invariance of spherical lightwaves. However the ``slope" of $\begin{pmatrix} 1 \\ 1\end{pmatrix}$ and $$a_\theta.\begin{pmatrix} 1 \\ 1\end{pmatrix}=\begin{pmatrix} \cosh \theta+\sinh \theta \\ \cosh \theta+\sinh \theta \end{pmatrix}$$ is identically equal to $c=1$ in accordance with (A2).

To further illustrate our earlier comments on the nonexistence of canonical parameters on $N$, consider that for any $C^1$ monotonic function $f: \bR \to \bR$ we obtain a curve $\epsilon(t)=\epsilon_f(t)$ $=\begin{pmatrix} f(t) \\ f(t)   \end{pmatrix}=f(t) \begin{pmatrix} 1 \\ 1 \end{pmatrix}$ for $t\in \bR$. Then $\epsilon_f(t)$ is supported on a straight line in $N$, and the ``photon" $\epsilon_f$ can be said to propagate in a straight line with constant speed $c=1$. It might be said that $\epsilon_f$ is \emph{not uniform} in the parameter $t$, but we argue that there exists no \emph{Lorentz invariant} definition of ``uniform parameter", especially on the null cone. 
 %This is because we do not identify the variational equation $\delta \int ds=0$ with the Riemannian geodesic equation $\nabla_{\gamma'} \gamma' =0$.

%The curve $\begin{pmatrix} \cos(t) \\ \cos(t) \end{pmatrix}$ might appear to satisfy (A2) except for the evident change in direction at
% as the more conventional curve with $f(t)=t$ and $\begin{pmatrix} t \\ t \end{pmatrix}$. In this sense we argue that the propagation of light, and specifically it's velocity, is underdetermined by (A2). 



%In other words the equation $x^2 = c^2 t^2$ is not the equation of sphere when $x,t$ are variable, and likewise $\xi^2=c^2 \tau^2$ is not the equation of sphere when $\xi,\tau$ are both variable and nonconstant. 
%Thus we find a gap in Einstein's purported proof. %The curious reader may consult Einstein's own words, e.g. \cite[Ch.11, pp.39]{einstein2019relativity} and verify that Einstein does not treat the general case but restricts himself to a velocity parallel to $x$-axis.  

%Moreover the algebraic expression \begin{equation}\label{a}\xi^2+\eta^2+\zeta^2=c^2 \tau^2\end{equation} is presumed to represent the ``equation of a sphere". However, the algebraic expression \eqref{a} of four independant variables $\xi, \eta, \zeta,\tau$ does \emph{not} represent a geometric sphere unless, e.g., $$\del \tau / \del \xi=0, ~ \del \tau / \del \eta=0, \del \tau / \del \zeta=0$$ for every $\xi, \eta, \zeta, \tau$ satisfying \eqref{a}. As the above two-dimensional case illustrates, this is not the case.

%Thus we find a positive gap in Einstein's argument, c.f. \cite{bryant}, \cite{crothers}. 
\section{Radius is Not a Lorentz Invariant Variable}\label{wes}
%\subsection{Homogeneous Wave Equation and Radial Solutions}

At this stage in our essay we now consider the wave interpretation in more detail. Our first step is to demonstrate a simple relationship between the Lorentz invariance of Minkowski's $ds^2$ \eqref{mform} and d'Alembert's operator $\square$. Informally we view $\square$ as ``dual" to $ds^2$ in the following sense. In \S\ref{s3} we referred to the results of \cite{elton2010indefinite}, \cite{arminjon2018lorentz} on the uniqueness of Lorentz invariant quadratic forms modulo homothety. Their same proof implies the following:

\begin{lem}
Let $C$ be the algebra of polynomial functions on $\bR^{1,3}$ and the contragredient representation $\rho^*$ of the Lorentz group. Then d'Alembert's wave operator $\square$ is the unique Lorentz invariant second order linear operator on $C$ modulo homothety.
\end{lem}

Therefore Einstein's assertion (i) that Lorentz transformations define the change-of-variables formulae for inertial observers $K, K'$ also implies that d'Alembert's operator $\square$ is essentially the unique second order operator defined simultaneously for all inertial observers. 

The Lorentz invariance of $\square$ implies the solutions of the homogeneous wave equation (HWE) are Lorentz covariant. So if $\phi=\phi(x,y,z,t)$ is a regular function satisfying (HWE)
\begin{equation} \label{hw}
\phi_{xx}+\phi_{yy}+\phi_{zz}-\frac{1}{c^2}\phi_{tt}=0.
\end{equation} and if $\lambda$ is a Lorentz transform with $(\xi, \eta, \zeta, \tau)=\lambda \cdot (x,y,z,t)$, then $$\phi':=\lambda^*(\phi)=\phi\circ\lambda^{-1}$$ is a solution of the homogeneous wave equation $$\phi_{\xi \xi}+\phi_{\eta \eta}+\phi_{\zeta \zeta}-\frac{1}{c^2}\phi_{\tau \tau}=0.$$ This is verified by elementary computation, substituting the formulae for the Lorentz transform. 

The main point we want to emphasize is this: \textbf{the set of \emph{radial solutions} of \eqref{hw} is \emph{not} Lorentz invariant}. Indeed the spatial radius variable $r^2=x^2+y^2+z^2$ is not a Lorentz invariant variable. This was discussed in \S\ref{einsteinproof}, but we give another explanation in terms of Lie groups below. In group theoretic terms, a solution $\phi=\phi(x,t)$ of \eqref{hw} is said to be \emph{radial by an observer $K$} iff $\phi(Ax,t)=\phi(x,t)$ for every rigid motion $A\in SO(3)$ in the space variables $x,y,z$ of $K$. Implicitly this requires a Lie group representation $\rho$ of $SO(3)$ into the Lorentz group $G\simeq O(3,1)$. But this choice of maximal compact subgroup is noncanonical. Different inertial observers $K, K'$ generally choose different orthogonal symmetry groups, e.g. by their own ``physical sum of squares" formula $\xi^2+\eta^2+\zeta^2$. Of course the Minkowski element \eqref{a} is invariant and canonical, but any decomposition into ``spatial" and ``time" requires arbitrary choices by the observer, and these choices are not Lorentz invariant. For example, while the open set of timelike vectors $\{h(v)<0\}$ is invariantly defined, there is no Lorentz invariant choice of timelike vector. Likewise among the spacelike set $\{h(v)>0\}$ there is no Lorentz invariant choice of orthogonal three-dimensional frame. 

Furthermore, for the motion of light pulses according to (A12) the Minkowski element vanishes identically, and the only canonical tensor element becomes the constant zero element. After a Lorentz change of variables we find a new solution $\phi'$ as above, but this solution need not be radial in the inertial frame $K'$. Indeed the rigid $K$-space motion $A$ will not generally preserve the space coordinates $\xi, \eta, \zeta$ of $K'$. Thus we find $A$-motions nontrivially depend on the $K'$-time variable $\tau$. This again reflects the nonexistence of a Lorentz invariant radius.

%Change of variables formula would require the $G$-conjugate $\rho^g$, $g\in G$, 







\section{Some Objections and Responses}\label{objections}
Given the critical nature of this article, we here respond to some potential objections. First, one might object that all our arguments reduce to the observation that spheres in the frame $K$ are transformed to ellipsoids in $K'$ as is well-known \cite[\S 4]{einstein1905electrodynamics}. But we remind the reader that Lorentz contraction is assumed to affect \emph{material} objects, even independantly of the nature of the material. So material spheres in $K$ become material ellipsoids in $K'$, where the eccentricity of the $K'$-ellipsoid is nontrivial and independant of the material nature of the sphere. But we respond that light spheres are \emph{immaterial} and not themselves subject to Lorentz contraction. In fact if light spheres were subject to the same effects as material spheres, then (A2) would definitely be false.

 %Evenmore (A2) clearly requires light waves be spherical in every reference frame if they are spherical in one reference frame.

Second, critics may object that (A12) only requires the consistent \emph{measurement} of $c$ in arbitrary reference frames $K, K'$. This would replace the formal ``law" (A2) with some rule of thumb for measurements. But this immediately leads to a well-known experimental difficulty at the core of special relativity, namely the impossibility of measuring the \emph{one-way} speed of light. For space and time measurements are always dependant on material objects and often non local, having sources and receivers separated by large distances. The impossibility of synchronizing non local clocks leads to the impossibility of measuring the one-way velocity of light. That all measurements of $c$ only succeed in measuring the ``two-way" or ``round-trip" velocities of light where source and receiver coincide is discussed in \cite{zhang1997special}, \cite{israel}. See also \cite{vid}. Moreover in studying the two-way velocity of light, one needs further postulate that the velocity $c$ is constant (uniform) throughout its two-way journey, as Einstein argued \cite[Ch.8]{einstein2019relativity}. But this assumption is arbitrary and unverifiable. %This article argues that the incompatibility of (A12) is not merely apparent, but \emph{essential} evidence that (A2) is not the correct natural law for (A2) and is not subject to measurement.

Third, the interesting textbook \cite[pp.8-10, 21-22]{rindler} admittedly attempts 
\begin{quote}
\emph{``in spite of its historical and heuristic importance, $\ldots$ to de-emphasize the logical role of the law of light propagation [(A2)] as a pillar of special relativity."}
\end{quote}
Rindler claims that
\begin{quote}
\emph{``a second axiom [(A2)] is needed \emph{only} to determine the value of a constant $c$ of the dimensions of a velocity that occurs naturally in the theory. But this could come from any number of branches of physics -- we need only think of the energy formula $E=mc^2$, or de Broglie's velocity relation $u v =c^2$." }
\end{quote}
Rindler's objection is interesting, and our response is simply that the above quoted formulas are \emph{equivalent} to (A2), and not independant in any logical or physical sense. The constant $c$ is of course central to physics, and $c$ was first formulated and estimated by Wilhelm Weber circa 1846, and even before J.C. Maxwell's famous treatise. Weber further studied $c$ with G. Kirchoff in the telegraphy equations. C.f. \cite{assis1999meaning} and \cite{awk}. Here $c$ is the velocity of an electric signal propagated through a wire of arbitrarily small resistance. But observe that the Weber-Kirchoff definition of $c$ is not equivalent to the $c$ of Einstein's special relativity: Einstein defines $c$ as the velocity in vacuum, and Weber-Kirchoff define $c$ as velocity of signal propagation in a \emph{material} wire! So all formulas involving $c$ are based essentially on some form of (A2), and the logical pillar remains unmoved. In otherwords, there does not appear any independant relation involving $c$ \emph{en vacuo} apart from Einstein's (A2). 

%See \cite{assis1999meaning} and \cite{awk} for further references and discussion. Even the incredible Weber-Kohlrausch formula expressing $c$ as ratio of electric and magnetic dielectric constants presumes a material medium, i.e. the ratio is undefined in vacuum. While an independant law involving $c$ (velocity in vacuum) could potentially serve as a logical substitute for (A2), this remains strictly hypothetical since no such formula appears to exist -- 

A fourth objection might criticize our argument for not properly accounting for the so-called wave-particle duality of light, e.g. ``Bohr's complementarity". Our article treats both cases (corpuscular and undulatory) showing that (A12) is underdetermined in \emph{both} cases. In section \ref{wes} we observed that ``radial solutions" of the homogeneous wave equation do not constitute a Lorentz invariant set: there does not exist solutions $\phi$ of the wave equation which are radial in every inertial frame. The photon theory is addressed in \S \ref{s3}. The incompatibility of (A12) with \emph{both} the wave and particle model has been highlighted by A.K.T. Assis \cite[\S 7.2.4, pp.133]{assis1999relational}:  
\begin{quote} \emph{``we can only conclude that for Einstein the velocity of light is constant not only whatever the state of motion of the emitting body [source], but also whatever the state of motion of the receiving body (detector) and of the observer."} 
\end{quote}
For waves in physical medium, the velocity of emission is independant of the velocity of the source, since waves are transmitted \emph{by} the medium and their velocities a property \emph{of} the medium. Furthermore for both particles and waves, it is known that velocity is dependant on the velocity of the receiver. According to (A2), light is postulated to exhibit properties unlike both waves and particles. Indeed we argue that (A2) \emph{contradicts} the supposed complementarity and wave-particle duality, i.e.  (A2) requires light to behave contrary to \emph{both} the wave and particle interpretations. We refer the reader to Assis' work for further details [Ibid].








%A popular fifth objection is that special relativity has apparently been experimentally verified, ad nauseum, a typical example being the clocks on GPS satellites. However the synchronization of such clocks essentially depends on Newtonian calculations, especially the expected time of arrival $t=d/c$, where $d$ is the distance.  


%Fourth, there is very important observation of A.K.T. Assis that even the so-called wave-particle duality interpretation is untenable in the following sense:


%For indeed Einstein derives $E=mc^2$ from a type of ``kinetic energy" formula for massless light pulses [ref].

%Our response to this third objection is that the above formulas cannot be (and never have been) so used to identify $c^2$ as a velocity. A single formula between three variables, e.g. $E,m,c$ or $u,v,c$, cannot be used to define a variable until the remaining two variables are defined and computable. 

%Our response is that the above quoted formulas are equivalent to Weber Kroschauer formula relating electric and magnetic permittivity.

%So we rebut Rindler's third objection by indicating that apparently ``light" has a definite law, and all fundamental formulas involving light are physically equivalent to (A2) or some weaker form. 


\section{Ralph Sansbury's Experiment}\label{sansbury}

Is it possible that \textbf{light is not \emph{something} that travels through space?}. This was proposed by Ralph Sansbury \cite{sansburylight}, and the following experiment is quoted in full from R. Sansbury's book \cite{sansburyspeed}. Recall that $c$ is well approximated at $1$ foot per nanosecond.

\begin{quote}
\emph{ (Case 1) A $15$ nanosecond light pulse from a laser was sent to a light detector, $30$ feet away. When the light pulse was blocked at the photodiode during the time of emission, but unblocked at the expected time of arrival, $31.2$ nanoseconds after the beginning of the time of emission, for $15$ nanosecond duration, little light was received. (A little more than the $4mV$ noise on the oscilloscope). This process was repeated thousands of times per second.}

\emph{(Case 2) When the light was unblocked at the photodiode during the time of emission ($15$ nanoseconds) but blocked after the beginning of the time of emission, during the expected time of arrival for $15$ nanoseconds, twice as much light was received ($8mV$). This process was repeated thousands of times per second.}

\emph{This indicated that light is not a moving wave or photon, but rather the cumulative effect of instantaneous forces at a distance. That is, undetectable oscillations of charge can occur in the atomic nuclei of the photodiode that spill over as detectable oscillations of electrons after a delay.}
\end{quote}

This important experiment has apparently not been repeated, despite it's simplicity. We refer the reader to R. Sansbury's book [Ibid] for further details and explanation via his \emph{cumulative instantaneous action at a distance} theory of light. Discussions on Sansbury's experiment typically argue that one needs the ability to block and unblock the photodiode at 15 nanosecond intervals, and that this is below the speed of so-called ``Pockel cells." A modification of Fizeau's tooth wheel experiment might render an experimental apparatus similar to R. Sansbury's setup.

%One distinction between Fizeau and Sansbury's experiment,is that Fizeau's is a two-way experiment, while Sansbury's is a one-way. In otherwords, its potentially very difficult to synchronize time with the source and receiver, since they are assumed to be physically separated, unlike Fizeau's experiments which are two-way. There is a two-way mirror splitting at the source in Fizeau's experiment which also reflects light to an observer lens. Discussions on Sansbury's experiment typically argue that one needs the ability to block and unblock the photodiode at 15 nanosecond intervals, and that this is below the speed of so-called ``Pockel cells." 


% The author's opinion is that Sansbury's ideas combined with W. Weber's work is a promising path of research which needs be investigated.

%Remark. Dr. R. Sansbury is deceased


%============================
%\section{Conclusion}
%This article attempts to lay out in plain mathematical terms various hazards and errors in the foundations of special relativity. The issues can subtle and easily overlooked. But these errors possibly stem from a deeper source, namely the \emph{assumption} that light needs be \emph{something that travels} through vacuum. From this author's perspective, this assumption still remains to be experimentally proven, and this is why we have included R. Sansbury's speculative experiment. 
%============================



%The experimental impossibility of measuring the one-way speed of light is further evidence that the basic assumptions of special relativity are nonphysical. 

%Thus the possibility of persistent errors in the foundations of special relativity, is that too few persons learn the fundamentals, and revisit them over time, and are able to discern the extremely subtle errors. 


%\section{Conclusion}
%This article attempts to outline inconsistencies in the foundations of Einstein's special relativity, and specifically Einstein's attempted resolution of the apparent incompatibility between (A1) and (A2). Einstein's argument suffers from two defects, as we outlined in Section \ref{li}.













%The possibility of such a fundamental error being unnoticed for more than one hundred years appears very po


%The consequences of the Michelson-Morley's null result









%Thus observation strictly supports only a weaker form (A2w) which might be formulated as follows.

%Indeed special relativity forbids the possibility of having two ``synchronized" clocks at distinct spatial positions; only ``local times" are permitted. 

%It is useful to recall an observation of F.K.T. Assis that material pendulums are even unreliable for nonlocal measurements because their oscillation periods are affected by their lengths and gravitational force [include formula]. N.B. the pendulum demonstrates time dilation without any Lorentz contraction. 
%\begin{itemize}
%\item[(A2w)] 
%The average velocity of every closed light trajectory, where initial and final spatial position are identical, is constant in vacuum.
%\end{itemize}

%Einstein chose to further assume that the velocity is everywhere constant along its trajectory. Admittedly this is a supposition, as he writes: 

%The convention that light travels at constant velocity $c$ during every round-trip means (A2) is replaced by (A2'): 

%And presumably the measurements that any observer $K$ might perform to test (A2) are nonlocal. Specifically, the clock by which the observer might reckon the ``velocity" of a light pulse is required to be stationary, fixed in place at the origin, e.g. Fizeau's experiments [ref], [?]. If we limit ourselves to local measurements, then the critics ought to replace (A2) with the further assumption ``that light also propagates ...". 

%But velocity is ex definitio the ratio of \emph{distance} over \emph{time}. It's possible (in logical sense) that the combination of length contraction and time dilation might yield the same measurement for all observers $K, K'$. However the material properties of both rulers and clocks make these measurements untenable. For a history of this controversy, the reader might investigate the problem of the so-called ``one-way" measurement of the speed of light. See \cite{vid} for an amusing presentation. 




%The above gap in Einstein's argument has been investigated by several critical authors, notably \cite{bryant}, \cite{crothers}. Our goal in this article has been to describe this gap in simple mathematical terms. 

%For Einstein, the purpose of his proof was to demonstrate that his so-called law of the propogation of light was indeed compatible with the principle of restricted relativity. 


% -- however it does not and a definite gap remains. 



%If Einstein's Special Relativity be falsified, where does physics stand today? This author reckons that Weber-Amp\`ere's relational electrodynamics as developed by \cite{assis1}, \cite{aw} can restore the 21st century to the correct path of investigation, and which already contains promising germs of unifying gravitation and strong nuclear forces in electrodynamics.



%For material motions the apparent incompatibilities can be resolved by postulating length contraction and time dilations in the directions of uniform velocity. However for the case of light pulses, we find the gap in Einstein's proof is an essential gap between (i) and (ii) which even Lorentz transformations do not resolve. 



%But what are the consequences of such a finding? According to Einstein, the principles of special relativity begin with Galilean relativity, namely the invariance of the laws of kinematics under affine transformations. It is apparently an experimental fact, that we can all discover, that the Laws of physics are the same in all nonaccelerated reference frames. Einstein calls this the Principle of Restricted Relativity: If $K'$ is a coordinate system moving uniformly (and devoid of rotation) with respect to a coordinate system $K$, then natural phenomena run their course with respect to $K'$ according to exactly the same laws as with respect to $K$. 

%Next Einstein claims  In view of the addition of velocities of particles, there is indeed an apparent incompatibility of the principle and the so-called law. 

%Einstein claims this apparent incompatibility can be resolved by Lorentz transformation law, i.e. by postulating length contraction and time dilation in the direction of uniform velocity. However in the case of the light pulses, the subject turns from material in motion to the transmission of light, which is hypothesized to be something that travels through space.




%[Examples: Organ orthogonal vs parallel to its direction of uniform motion has same pitch] [Examples??]



%I don't recall the ``speed of light being a finite constant" as a law of my youth. It was hardly mentioned as i recall. But i did wonder and stand mystified by the sun's rising and setting, moons, ...


%But there is another postulate, which Einstein adds, namely the purpotedly constant finite velocity of light (en vacuo). He even suggests that this is apparent to school children, which is doubtful. That the transmission of light was not instantaneous had been proposed according to Galileo's rough experiments (lamps at a great distance), and more strongly by Cassini/Roemer and the transits of Io across Jupiter. Thus E adds a second postulate: constant velocity of propogation of light. 

%The solution to the apparent contradiction is, according to Einstein, to modify the idea that ``distance" between two points on a rigid body is independant of the uniform motion of the reference body [ref:E, Ch 11, pp.34]. %Now Einstein clearly indicated the apparent contradiction of these two postulates, by comparing two objects in constant relative motion. For instance, a train in uniform motion relative to a fixed embankment. 

%Einstein claimed to have resolved this apparent contradiction by Lorentz transformations. I.e. by postulating length contraction and time dilation in the direction of uniform velocity, c.f. \cite{einstein2019relativity}[Ch.11, pp.39]. The reader can verify that Einstein does not treat the general case, but restricts himself to a velocity parallel to $x$-axis. Unfortunately a more detailed demonstration was left to future more critical authors, e.g. \cite{crothers}, \cite{bryant}, and nearly one hundred years expired before the gap was noticed.



% leads to the constant velocity in all inertial reference frames. However does the reader sense an inherent tension here? Length contraction is a function of material objects moving near speed of light, but not necessarily involves the contraction of light waves themselves.



%It is unfortunate that E did not give detailed demonstration, because therein he would have been disproven! But a mathematician who does not perform his calculations is like a physicist who does not perform his experiments, lest he be disproven! 

%So let us now begin the formal details: E needs formally demonstrate that the equation $r=ct$ in the $K$ frame is mapped onto $r'=c\tau$ in the $K'$ frame. In particular, as we vary over the points $p$ of the sphere $r=ct$, we need establish that the Lorentz image $p'=Lp$ under the Lorentz transformation $L$ describes a sphere in the $K'$ frame. In otherwords, and here we need be cautious, we need establish that the image radius $r'$ is constant with respect to points $p$ satisfying $r=ct$.

%There is a further error in Einstein: an erroneuous ``factorization" statement of $L=B\circ R$, where $B$ is a \emph{boost} parallel to $x$-axis, and $R$ is a \emph{spatial rotation}. [But the correct factorization is: XX]

%Lorentz group= boost along x-axis, composed with all space rotations.[error?]

%As specified by [ref], Einstein's two conditions are: 

%(I) a spherical equation is satisfied $$(x-x_0)^2+...=R^2$$

%(II) the above equation in $x,y,z,R$ defines a sphere if $R$ is constant as $x,y,z$ vary. 



%Einstein is correct that the image of $R=ct$ under Lorentz group has the form of a ``spherical equation", namely $\xi^2+\eta^2+\zeta^2=c^2 \tau^2$; however the equation represents a sphere(!) only if the ``radius" is constant, i.e. the equation holds and both LHS and RHS are constant. 

%Consider the basic Lorentz boost along the $x$-axis at a velocity $v$. The Lorentz change of coordinates is $$\xi=\gamma (x-vt), \eta=y, \zeta=z, \tau=\gamma(t-vx/c^2), $$ where $\gamma$ is the Lorentz factor. We assume $x,y,z,$ are constrained to the sphere $(x-x_0)^2+..=R^2$.

%Now Einstein is correct that we have $$x^2+y^2+...=c^2 t^2$$ and also $$\xi^2+..=c^2 \tau^2,$$ according to the Lorentz transformation.

%If $t,\tau$ are constant, then both equations represent spheres. However, and this is key point, we observe that $\tau$ is not constant with respect to $x,y,z$. This is made evident by the following computation, which [E] neglects to perform in [ref]: namely $dR'/dx$ is nonzero, even when $x$ is constrained to the sphere $R=ct$. This is key computation which [E] neglects to perform.

%By chain rule, $$\frac{dR'}{dx}=\frac{dR'}{d\tau} \frac{d\tau}{dx}+\frac{dR'}{d\zeta} \frac{d\zeta}{dx}\frac{dR'}{d\nu} \frac{d\nu}{dx}\frac{dR'}{d\xi} \frac{d\xi}{dx},$$ and which according to Lorentz transformation [ref], is equal to $-(v/c)\gamma$. Since this expression is decidedly nonzero, it follows that the radius $R'$ is non constant. 

%Counter arguments: (i) aspects of relativity, such as simultaneity, length contraction, time dilation, are already experimentally valid. 

%(ii) the Lorentz image of the sphere is the view relative an observer in $K$ (and which image is not a sphere, but elliptical), whereas the Lorentz image is ``correct" (i.e. spherical) for a $K'$ observer. 



\printbibliography[title={References}]
\end{document}
