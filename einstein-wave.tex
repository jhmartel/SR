\documentclass[12pt]{amsart}
\usepackage{mathrsfs}
\usepackage{amsmath}
\usepackage{amssymb}
\usepackage{amsfonts}
\usepackage{amsopn}
\usepackage{amsthm}
\usepackage{latexsym}
\usepackage[all]{xy}
\usepackage{enumerate}
\usepackage{geometry}
%\usepackage{biblatex}
%\usepackage{hyperref}
%\usepackage[autostyle]{csquotes}
\usepackage{fancyhdr}
\usepackage{graphicx}
\usepackage{wrapfig}
\usepackage{float}

\usepackage[
    backend=biber,
style=alphabetic,
sorting=nyt,
   % style=authoryear-icomp,
    %sortlocale=de_DE,
    %natbib=true,
    %author=true,
%style=verbose,
%journal=true,
%url=true, 
%    doi=false,
%    eprint=true
]{biblatex}
\addbibresource{biblio.bib}

\usepackage[]{hyperref}
\hypersetup{
    colorlinks=true,
}


\newtheorem{thm}{Theorem}
\newtheorem*{half}{Halfspace Condition}
\newtheorem{lem}[thm]{Lemma}
\newtheorem{prop}[thm]{Proposition}
\newtheorem*{prob}{Problem}
\newtheorem{cor}[thm]{Corollary}
\newtheorem{question}[thm]{Question}
\newtheorem*{flashbang}{FlashBang Principle}
\newtheorem*{Lefschetz}{Lefschetz theorem}
\newtheorem*{hyp}{Hypothesis}
\theoremstyle{definition}
\newtheorem{dfn}[thm]{Definition}
\newtheorem{exx}[thm]{Example}
\theoremstyle{remark}
\newtheorem{rem}[thm]{Remark}
\newcommand{\fh}{\mathfrak{h}}
\newcommand{\bn}{\mathbf{n}}
\newcommand{\bC}{\mathbb{C}}
\newcommand{\bG}{\mathbb{G}}
\newcommand{\bR}{\mathbb{R}}
\newcommand{\fB}{\mathfrak{B}}
\newcommand{\bZ}{\mathbb{Z}}
\newcommand{\bQ}{\mathbb{Q}}
\newcommand{\bq}{\bar{Q}[t]}
\newcommand{\bb}{\bullet}
\newcommand{\del}{\partial}
\newcommand{\sB}{\mathscr{B}}
\newcommand{\sE}{\mathscr{E}}
\newcommand{\mR}{\bR^\times_{>0}}
\newcommand{\conv}{\overbar{conv}}
\newcommand{\sub}{\del^c \psi^c(x')}
\newcommand{\subb}{\del^c \psi^c(x'')}
\newcommand{\hh}{\hookleftarrow}
\newcommand{\bD}{\mathbb{D}}
\newcommand{\Gm}{\mathbb{G}_m}
\newcommand{\uF}{\underline{F}}
\newcommand{\sC}{\mathscr{C}}
\newcommand{\bX}{\overline{X}^{BS/\bQ}}
\newcommand{\sT}{\mathscr{T}}
\newcommand{\sW}{\mathscr{W}}
\newcommand{\sZ}{\mathscr{Z}}

\begin{document}

\title{On Einstein-Lorentz' Alleged Proof of Spherical Invariance of Light Pulses in Special Relativity}
\author{J. H. Martel}
\date{\today}

\maketitle

The subject of this brief note is controversial. We observe the Lorentz $SO(3,1)$-invariance of the quadratic form $h=x^2+y^2+z^2-c^2t^2$ in no way implies the Lorentz invariance of the forms $h_1=x^2+y^2+z^2$ and $h_2=c^2 t^2$. As corollary we find Einstein's proposed resolution of the apparent incompatibility of the principle of restricted relativity and the purported law of the constancy of light invalidated by basic computations, c.f. \cite[Ch.7, 11]{einstein2019relativity}. The argument is general and applies to split orthogonal groups of arbitrary rank $SO(1,1)$, $SO(2,1)$, etc.. %In the most basic case of $SO(1,1)$, the key observation is that a quadratic expression like $x^2=r^2$ in the $xr$-variables does not represent a ``sphere"; the expression represents a sphere of radius $r$ and only if the additional equation $\del x /\del r =0$ holds for every $x$. But the expression $\del x / \del r$ is not a tensor, i.e. satisfies no invariant 

%\section{}
As far as mathematics is concerned, here is the issue in simplest terms. Minkowski's form $h=x^2+y^2+z^2-c^2t^2$ is invariant with respect to Lorentz transformations, i.e. the Lie group $G=SO(h)=SO(3,1)$ where invariance means $\xi^2+\eta^2+\zeta^2-c^2 \tau^2$ is equal to $x^2+y^2+z^2-c^2 t^2$ for every Lorentz transform $(\xi, \eta, \zeta, \tau)$ of $(x,y,z,t)$. That is, the Minkowski form is a real valued function, and is invariant under Lorentz transformations. Now as Einstein's argument concerns the geometry of a wave front generated by a light pulse, i.e. the action of the Lorentz group on (subsets of) the so-called null cone $N=\{h=0\}$, where obviously $N$ is $G$-invariant and defined by the equation $x^2+y^2+z^2=c^2 t^2$. Next comes the key claim, that $N$ has no nonempty proper $G$-invariant subsets. More specifically the Lorentz invariance of $h, N$ in no way implies the Lorentz invariance of the forms $h'=x^2+y^2+z^2$ and $h''=c^2 t^2$, even despite their difference $h=h'-h''$ being invariant. 

The two-dimensional case in $xt$-variables is illustrative, where $h=x^2-c^2 t^2$ is invariant with respect to the group $G=SO(1,1)$ generated by $a_\theta:=\begin{pmatrix} \cosh \theta & \sinh \theta \\
\sinh \theta & \cosh \theta
\end{pmatrix}$, where $\theta\in \bR$. We see $SO(1,1)$ is isomorphic to the split multiplicative torus $\bR^\times _{>0}$ using the logarithm. The unit sphere includes two vectors $$\langle 1, 1 \rangle,~~ \langle -1,1\rangle,$$ and which are mapped by $a_\theta \in SO(1,1) \simeq \bR^\times_{>0}$ to $$\langle \xi,\tau\rangle=\langle \cosh \theta+\sinh \theta, \cosh \theta+\sinh \theta \rangle,~~ \langle -\cosh \theta+\sinh \theta, \cosh \theta-\sinh \theta \rangle.$$ But evidently $\xi^2 \neq x^2=1$ and $\tau^2 \neq t^2=1$ when $\theta\neq 0$. That is, the image of the unit sphere does not correspond to a sphere in $\xi \tau$ coordinates. In other words the equation $x^2 = c^2 t^2$ is not the equation of sphere when $x,t$ are variable, and likewise $\xi^2=c^2 \tau^2$ is not the equation of sphere when $\xi,\tau$ are both variable and nonconstant. 


%\section{}
The author submits that the above argument positively demonstrates a gap in Einstein's purported proof, c.f. \cite{einstein2019relativity}[Ch.11, pp.39] where the reader can verify that Einstein does not treat the general case but restricts himself to a velocity parallel to $x$-axis. Nearly one hundred years expired before the gap was (re)discoverd by more critical authors \cite{bryant}, \cite{crothers}. What are the consequences of such a gap? For Einstein, the purpose of his proof was to demonstrate that his so-called law of the propogation of light was indeed compatible with the principle of restricted relativity. The fundamental assumptions of Einstein's special relativity are (i) that the Laws of physics are the same in all nonaccelerated reference frames, where if $K'$ is a coordinate system moving uniformly (and devoid of rotation) with respect to a coordinate system $K$, then natural phenomena run their course with respect to $K'$ according to exactly the same laws as with respect to $K$. Next Einstein claims ``\emph{There is hardly a simpler law in physics than that according to which light is propograted in empty space. Every child at school knows, or believes he knows, that this propogation takes place in straight lines with velocity $c=300 000 km/sec$. \ldots Who would imagine that this simple law has plunged the conscientiously thoughtful physicist into the greatest intellectual difficulties?}" Thus in view of Fizeau's Addition of Velocities in classical mechanics, apparently (i) and (ii) are incompatible. Einstein claims this apparent incompatibility is resolved by Lorentz transformations -- however it does not and a definite gap remains. For material motions, the null result of Michelson-Morley is compatible with the postulates of length contraction and time dilations, as consistent with the Lorentz group. But for light pulses -- which is the subject of (ii) -- the Lorentz group is not consistent with (ii).

%If Einstein's Special Relativity be falsified, where does physics stand today? This author reckons that Weber-Amp\`ere's relational electrodynamics as developed by \cite{assis1}, \cite{aw} can restore the 21st century to the correct path of investigation, and which already contains promising germs of unifying gravitation and strong nuclear forces in electrodynamics.



%For material motions the apparent incompatibilities can be resolved by postulating length contraction and time dilations in the directions of uniform velocity. However for the case of light pulses, we find the gap in Einstein's proof is an essential gap between (i) and (ii) which even Lorentz transformations do not resolve. 



%But what are the consequences of such a finding? According to Einstein, the principles of special relativity begin with Galilean relativity, namely the invariance of the laws of kinematics under affine transformations. It is apparently an experimental fact, that we can all discover, that the Laws of physics are the same in all nonaccelerated reference frames. Einstein calls this the Principle of Restricted Relativity: If $K'$ is a coordinate system moving uniformly (and devoid of rotation) with respect to a coordinate system $K$, then natural phenomena run their course with respect to $K'$ according to exactly the same laws as with respect to $K$. 

%Next Einstein claims  In view of the addition of velocities of particles, there is indeed an apparent incompatibility of the principle and the so-called law. 

%Einstein claims this apparent incompatibility can be resolved by Lorentz transformation law, i.e. by postulating length contraction and time dilation in the direction of uniform velocity. However in the case of the light pulses, the subject turns from material in motion to the transmission of light, which is hypothesized to be something that travels through space.




%[Examples: Organ orthogonal vs parallel to its direction of uniform motion has same pitch] [Examples??]



%I don't recall the ``speed of light being a finite constant" as a law of my youth. It was hardly mentioned as i recall. But i did wonder and stand mystified by the sun's rising and setting, moons, ...


%But there is another postulate, which Einstein adds, namely the purpotedly constant finite velocity of light (en vacuo). He even suggests that this is apparent to school children, which is doubtful. That the transmission of light was not instantaneous had been proposed according to Galileo's rough experiments (lamps at a great distance), and more strongly by Cassini/Roemer and the transits of Io across Jupiter. Thus E adds a second postulate: constant velocity of propogation of light. 

%The solution to the apparent contradiction is, according to Einstein, to modify the idea that ``distance" between two points on a rigid body is independant of the uniform motion of the reference body [ref:E, Ch 11, pp.34]. %Now Einstein clearly indicated the apparent contradiction of these two postulates, by comparing two objects in constant relative motion. For instance, a train in uniform motion relative to a fixed embankment. 

%Einstein claimed to have resolved this apparent contradiction by Lorentz transformations. I.e. by postulating length contraction and time dilation in the direction of uniform velocity, c.f. \cite{einstein2019relativity}[Ch.11, pp.39]. The reader can verify that Einstein does not treat the general case, but restricts himself to a velocity parallel to $x$-axis. Unfortunately a more detailed demonstration was left to future more critical authors, e.g. \cite{crothers}, \cite{bryant}, and nearly one hundred years expired before the gap was noticed.



% leads to the constant velocity in all inertial reference frames. However does the reader sense an inherent tension here? Length contraction is a function of material objects moving near speed of light, but not necessarily involves the contraction of light waves themselves.



%It is unfortunate that E did not give detailed demonstration, because therein he would have been disproven! But a mathematician who does not perform his calculations is like a physicist who does not perform his experiments, lest he be disproven! 

%So let us now begin the formal details: E needs formally demonstrate that the equation $r=ct$ in the $K$ frame is mapped onto $r'=c\tau$ in the $K'$ frame. In particular, as we vary over the points $p$ of the sphere $r=ct$, we need establish that the Lorentz image $p'=Lp$ under the Lorentz transformation $L$ describes a sphere in the $K'$ frame. In otherwords, and here we need be cautious, we need establish that the image radius $r'$ is constant with respect to points $p$ satisfying $r=ct$.

%There is a further error in Einstein: an erroneuous ``factorization" statement of $L=B\circ R$, where $B$ is a \emph{boost} parallel to $x$-axis, and $R$ is a \emph{spatial rotation}. [But the correct factorization is: XX]

%Lorentz group= boost along x-axis, composed with all space rotations.[error?]

%As specified by [ref], Einstein's two conditions are: 

%(I) a spherical equation is satisfied $$(x-x_0)^2+...=R^2$$

%(II) the above equation in $x,y,z,R$ defines a sphere if $R$ is constant as $x,y,z$ vary. 



%Einstein is correct that the image of $R=ct$ under Lorentz group has the form of a ``spherical equation", namely $\xi^2+\eta^2+\zeta^2=c^2 \tau^2$; however the equation represents a sphere(!) only if the ``radius" is constant, i.e. the equation holds and both LHS and RHS are constant. 

%Consider the basic Lorentz boost along the $x$-axis at a velocity $v$. The Lorentz change of coordinates is $$\xi=\gamma (x-vt), \eta=y, \zeta=z, \tau=\gamma(t-vx/c^2), $$ where $\gamma$ is the Lorentz factor. We assume $x,y,z,$ are constrained to the sphere $(x-x_0)^2+..=R^2$.

%Now Einstein is correct that we have $$x^2+y^2+...=c^2 t^2$$ and also $$\xi^2+..=c^2 \tau^2,$$ according to the Lorentz transformation.

%If $t,\tau$ are constant, then both equations represent spheres. However, and this is key point, we observe that $\tau$ is not constant with respect to $x,y,z$. This is made evident by the following computation, which [E] neglects to perform in [ref]: namely $dR'/dx$ is nonzero, even when $x$ is constrained to the sphere $R=ct$. This is key computation which [E] neglects to perform.

%By chain rule, $$\frac{dR'}{dx}=\frac{dR'}{d\tau} \frac{d\tau}{dx}+\frac{dR'}{d\zeta} \frac{d\zeta}{dx}\frac{dR'}{d\nu} \frac{d\nu}{dx}\frac{dR'}{d\xi} \frac{d\xi}{dx},$$ and which according to Lorentz transformation [ref], is equal to $-(v/c)\gamma$. Since this expression is decidedly nonzero, it follows that the radius $R'$ is non constant. 

%Counter arguments: (i) aspects of relativity, such as simultaneity, length contraction, time dilation, are already experimentally valid. 

%(ii) the Lorentz image of the sphere is the view relative an observer in $K$ (and which image is not a sphere, but elliptical), whereas the Lorentz image is ``correct" (i.e. spherical) for a $K'$ observer. 



\printbibliography[title={References}]
\end{document}
